% Copyleft 2010 by Walter Vargas <walter@covetel.com.ve>

\documentclass[9pt]{beamer}
\usepackage{listings}
\lstset{
    language=HTML 
}
\lstset{
    basicstyle=\scriptsize, 
    stringstyle=\ttfamily,
    showstringspaces=false, 
    numbers=left,
    numberstyle=\scriptsize,
    tabsize=2,
}

% Configuración de la apariencia. 

\usetheme{Szeged}
\usecolortheme{beaver}
\usefonttheme[onlylarge]{structurebold}
\setbeamerfont*{frametitle}{size=\normalsize,series=\bfseries}

\usepackage{color}
\definecolor{lightgray}{rgb}{.9, .9, .9}
\definecolor{darkgray}{rgb}{.4, .4, .4}
\definecolor{purple}{rgb}{0.65,  0.12,  0.82}
\definecolor{azulito}{HTML}{0066CC}

\lstdefinelanguage{HTML}{
keywords={xml, DOCTYPE, head,  body, html, return,  this}, 
keywordstyle=\color{azulito}\bfseries, 
ndkeywords={title, option, form, textarea, table, th, tr, td, p, br, input,
button, frame, iframe, div, optgroup, select, fieldset,  legend, h1, h2, h3,
h4, h5, h6, p, blockquote, address, em,strong, abbr, acronym, cite, dfn, code,
kbd, samp, var, q, ul, li, dl, dt, dd, strong, span, add, attr, is, prev,
parent, children, parents, gt, find, lt, eq, not,  contains, filter, visible,
hidden, nth, even, odd, css}, 
ndkeywordstyle=\color{blue}\bfseries, 
identifierstyle=\color{black}, 
sensitive=false, 
comment=[l]{//}, 
morecomment=[s]{/*}{*/}, 
stringstyle=\color{azulito}\ttfamily, 
morestring=[b]', 
morestring=[b]"
}

\lstset{
language=HTML, 
backgroundcolor=\color{lightgray}, 
extendedchars=true, 
basicstyle=\scriptsize\ttfamily, 
showstringspaces=false, 
showspaces=false, 
numbers=left, 
numbersstyle=\footnotesize, 
numbersep=9pt, 
tabsize=2, 
breaklines=true, 
showtabs=false, 
captionpos=b 
}


% Paquetes
\usepackage[utf8]{inputenc}
\usepackage[spanish]{babel}


% Titulo
\title[WebDesign] {Diseño Web | Tecnologías Involucradas}
\author[Walter Vargas]{ info@covetel.com.ve \inst{1}}
\subtitle{Capa de Presentación CSS}
\institute[covetel.com.ve]{ \inst{1} Cooperativa Venezolana de Tecnologías Libres R.S. }
\date

\begin{document}

\begin{frame} % (fold)
    \titlepage 
\end{frame}

\begin{frame} % (fold)
    \tableofcontents
\end{frame}

\section{JavaScript}

\subsection{Métodos de Implementación} % (fold)

% subsection Métodos de Implementación (end)

\begin{frame}[allowframebreaks, fragile]{Donde implementar JavaScript?} % (fold)

    JavaScript puede implementarse en una sola página o en todo un sitio. Como
    en CSS, puede incrustarse en un documento o estar fuera del documento. Los
    dos métodos se realizan con el elemento \texttt{<script>}
    Comenzaremos con el ejemplo incrustado:

        \begin{lstlisting}
<script type="text/javascript">
// <![CDATA[
...aqui va el codigo JavaScript
  // ]]>
</script>
        \end{lstlisting}

    Como se observa, el elemento \texttt{<script>} establece el bloque como si
    fuera un script y se ajusta al tipo MIME (utilizando el atributo
    \texttt{<type>}) a \texttt{<text/javascript>} (también se aceptaría
    \texttt{<text/ecmascript>}).

    El  \texttt{<// <!CDATA [ y // ]]>}) quizás le resulten familiares, ya que
    son una de las formas de designar comentarios en JavaScript e indican al
    script anterior que ignore el resto de cada una de esas líneas.

    Externalizar el JavaScript es el método preferido de implementación, porque
    permite incluir las mismas funciones en varias páginas (y puede evitar
    declarar el contenido como \texttt{<CDATA>}). Así se internaliza un
    script:

    \begin{lstlisting}

<script type="text/javascript" src="my_script.js"></script>

    \end{lstlisting}

    En este ejemplo se ha movido el script JavaScript a un archivo
    independiente y simplemente se ha incluido en el documento presentando su
    nombre de archivo como fuente \texttt{<src>} del elemento
    \texttt{<script>}. Puede incluir todos los scripts que quiera de este
    modo e incluso combinar este método con scripts incrustados como en este
    ejemplo de Google Analytics (www.google.com/analytics/):

    \begin{lstlisting}

<script src="http://www.google-analytics.com/urchin.js" type="text/javascript"></script>
<script type="text/javascript">
  // <![CDATA[
_uact = "UA-XXXXXXX-X";
urchinTracker();
  //   ]]>
</script>
    \end{lstlisting}

    El script externo reside en el servidor Google y es el mismo para todos los
    que utilicen Google Analytics. El script incrustado estable la cuenta de
    usuario (la variable \texttt{<_uacct>} y activa la función
    \texttt{<urchinTracker>}.

    Se recomienda mantener los elementos \texttt{<script>} en un área común en
    la cabecera \texttt{<head>} de las páginas \texttt{<(X)HTML>}. Es
    fundamentalmente por razones de convención y para facilitar el
    mantenimiento.

\end{frame}

\begin{frame}[allowframebreaks, fragile]{Sintaxis de JavaScript} % (fold)

% subsection Sintaxis de JavaScript(end)

\begin{frame}{fragile}
    
    Cada script consiste en una serie de declaraciones. Las declaraciones
    pueden terminarse con un salto de línea:

    \begin{lstlisting}
primera declaracion
segunda declaracion
    \end{lstlisting}

    ó con un punto y coma \texttt{<(;)>}

    \begin{lstlisting}
primera declaracion; segunda declaracion;
    \end{lstlisting}

    Para facilitar la legibilidad, y para evitar problemas de terminación de
    declaraciones, se recomienda usar ambos:

    \begin{lstlisting}
primera declaracion;
segunda declaracion;
    \end{lstlisting}

\end{frame}

\begin{frame}{fragile}[Comentarios] % (fold)

    JavaScript permite hacer comentarios en su código, pero de formas algo
    distintas. El primer estilo de comentario utiliza dos barras inclinadas:

    \begin{lstlisting}
// esto es un comentario
    \end{lstlisting}

    Este tipo de comentario hace que el intérprete ignore el resto de la línea.
    El segundo método le permite comentar a través de varias líneas:
   
   \begin{lstlisting}
/* esto es un comentario
   multilínea o de bloque */
    \end{lstlisting}

    Además de utilizar comentarios para dejar notas, son también muy útiles en
    los procesos de corrección de errores: puede dejar una línea o sección del
    código de un comentario para averiguar si es la que está dando el error.

\end{frame}

\begin{frame}{fragile}[Variables]
    
    En JavaScript se tiene que declarar las variables antes de comenzar a
    usarlas. Aunque es muy flexible para nombrar y declarar variables.
    Las variables se declaran utilizando la palabra clave reservada
    \texttt{<var>}. Los nombres de variable pueden tener cualquier extensión y
    contener números letras y ciertos caracteres no alfanuméricos. Conviene
    evitar los operadores aritméticos  \texttt{<+,-,*,/>} y comillas
    \texttt{<' y ">}. También tendrá que procurar que los nombres de variable
    no creen conflictos con palabras clases reservadas de JavaScript \texttt{<>} 


\end{frame}

\end{frame}

% subsection Diseños en dos columnas (end)

\end{document}
