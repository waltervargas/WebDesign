% Copyleft 2010 by Walter Vargas <walter@covetel.com.ve>

\documentclass[9pt]{beamer}
\usepackage{listings}
\lstset{
    language=HTML 
}
\lstset{
    basicstyle=\scriptsize, 
    stringstyle=\ttfamily,
    showstringspaces=false, 
    numbers=left,
    numberstyle=\scriptsize,
    tabsize=2,
}

% Configuración de la apariencia. 

\usetheme{Szeged}
\usecolortheme{beaver}
\usefonttheme[onlylarge]{structurebold}
\setbeamerfont*{frametitle}{size=\normalsize,series=\bfseries}

\usepackage{color}
\definecolor{lightgray}{rgb}{.9, .9, .9}
\definecolor{darkgray}{rgb}{.4, .4, .4}
\definecolor{purple}{rgb}{0.65,  0.12,  0.82}
\definecolor{azulito}{HTML}{0066CC}

\lstdefinelanguage{HTML}{
keywords={xml, DOCTYPE, head,  body, html, return,  this}, 
keywordstyle=\color{azulito}\bfseries, 
ndkeywords={title, option, form, textarea, table, th, tr, td, p, br, input,
button, frame, iframe, div, optgroup, select, fieldset,  legend, h1, h2, h3,
h4, h5, h6, p, blockquote, address, em,strong, abbr, acronym, cite, dfn, code,
kbd, samp, var, q, ul, li, dl, dt, dd, strong, span, add, attr, is, prev,
parent, children, parents, gt, find, lt, eq, not,  contains, filter, visible,
hidden, nth, even, odd, css}, 
ndkeywordstyle=\color{blue}\bfseries, 
identifierstyle=\color{black}, 
sensitive=false, 
comment=[l]{//}, 
morecomment=[s]{/*}{*/}, 
stringstyle=\color{azulito}\ttfamily, 
morestring=[b]', 
morestring=[b]"
}

\lstset{
language=HTML, 
backgroundcolor=\color{lightgray}, 
extendedchars=true, 
basicstyle=\scriptsize\ttfamily, 
showstringspaces=false, 
showspaces=false, 
numbers=left, 
numbersstyle=\footnotesize, 
numbersep=9pt, 
tabsize=2, 
breaklines=true, 
showtabs=false, 
captionpos=b 
}


% Paquetes
\usepackage[utf8]{inputenc}
\usepackage[spanish]{babel}


% Titulo
\title[WebDesign] {Diseño Web | Tecnologías Involucradas}
\author[Walter Vargas]{ info@covetel.com.ve \inst{1}}
\subtitle{JQuery}
\institute[covetel.com.ve]{ \inst{1} Cooperativa Venezolana de Tecnologías Libres R.S. }
\date

\begin{document}

\begin{frame} % (fold)
    \titlepage 
\end{frame}

\begin{frame} % (fold)
    \tableofcontents
\end{frame}

\section{Métodos de Efecto} % (fold)

\subsection{Efectos Pre-empaquetados} % (fold)

\begin{frame}[fragile]{Efectos Pre-empaquetados} % (fold)
\begin{itemize}
    \item .show([speed][, callback]) \textit{ speed= una cadena o un número para determinar cuánto tiempo se ejecutara la animación.}

    \textit{callback= una función para saber cuando ha finalizado la animación.}
    \begin{lstlisting}
    <div class="content">
      <div class="trigger button">Trigger</div>
      <div class="target"><img src="hat.gif" width="80" height="54"></div>
      <div class="long"></div>
    </div>
    \end{lstlisting}
    \begin{lstlisting}
    $('.trigger').click(function() {
      $('.target').show('slow',  function() {
        $(this).log('Effect Complete.');
      });
    });
    \end{lstlisting}
    \item .hide().
    \item .toggle().
    \item .slideDown() .slideUp() .slideToggle()
    \item .fadeIn() .fadeOut() .fadeTo()
\end{itemize}
\end{frame}

\subsection{Efectos Personalizados} % (fold)

\begin{frame}[fragile]{Efectos Personalizados} % (fold)
\begin{itemize}
    \item .animate(properties [, speed][, easing][, callback])  \textit{
    properties= un mapa de propiedades CSS por los que se moverá la animación.}

    \textit{ speed= una cadena o un número para determinar el tiempo que se ejecutara la animación.}

    \textit{ easing= una cadena facilita el uso de la función en la transición.}

    \textit{ callback= una función para saber cuando ha finalizado la animación.}
    \begin{lstlisting}
    <div class="content">
      <div class="trigger button">Trigger</div>
      <div class="target"><img src="hat.gif" width="80" height="54"></div>
      <div class="long"></div>
    </div>
    \end{lstlisting}
    \begin{lstlisting}
    $('.trigger').click(function () {
      $('.target').animate({
      'width': 300, 
      'left': 100, 
      'opacity': 0.25
      }, 'slow', function() {
        $(this).log('Effect Complete');
      });  
    });
    \end{lstlisting}
\end{itemize}
\end{frame}

\section{Métodos AJAX} % (fold)

\subsection{Interfaz de Bajo Nivel} % (fold)

\begin{frame}[fragile]{Interfaz de Bajo Nivel} % (fold)
\begin{itemize}
    \item .ajax(settings) \textit{ settings= un mapa que contiene las siguientes opciones para loas solicitudes:}
    \begin{itemize}
      \item url.
      \item dataType.
      \item timeout.
      \item error.
      \item success.
    \end{itemize}
    \begin{lstlisting}
    $.ajax({
      url: 'ajax/test.html', 
      success: function(data) {
        $('.result').html(data);
        $().log('Load was performed');
      }, 
    });
    \end{lstlisting}
    \item .ajaxSetup(settings) \textit{ settings= un mapa de opciones para
    solicitudes a futuro.} 
    \begin{lstlisting}
    $.ajaxSetup({
      url: 'ping.php', 
    });
    \end{lstlisting}
\end{itemize}
\end{frame}

\subsection{Métodos de Taquigrafía} % (fold)

\begin{frame}[fragile]{Métodos de Taquigrafía} % (fold)
\begin{itemize}
    \item .get(url[, data][, success]) \textit{ url= una cadena que contiene la URL donde la solicitud debe ser enviada.}

    \textit{ data= un mapa de datos enviados con la solicitud.}

    \textit{ success= una función que se ejecuta cuando la solicitud a sido procesada.}
    \begin{lstlisting}
    $.get('ajax/test.html', function (data) {
      $('.result').html(data);
      $().log('Load was performed');
    })
    \end{lstlisting}
    \item .getModified().
    \item .loadModified().
    \item .post().
\end{itemize}
\end{frame}

\begin{frame}[fragile]{Métodos de Taquigrafía} % (fold)
\begin{itemize}
    \item .load(url[, data][, success]) \textit{ url= una cadena que contiene la URL donde la solicitud debe ser enviada.}

    \textit{ data= un mapa de datos enviados con la solicitud.}

    \textit{ success= una función que se ejecuta cuando la solicitud a sido procesada.}
    \begin{lstlisting}
    $('.result').load('ajax/test.html');
    \end{lstlisting}
    \item .getJSON(url[, data][, success]) \textit{ url= una cadena que contiene la URL donde la solicitud debe ser enviada.}

    \textit{ data= un mapa de datos enviados con la solicitud.}

    \textit{ success= una función que se ejecuta cuando la solicitud a sido procesada.}
    \begin{lstlisting}
    $.getJSON('ajax/test.json', function(data) {
      $('.result').html('<p> + data.foo + '</p><p>' + data.baz[1] + '</p>');
      $().log('Load was performed');
    });  
    \end{lstlisting}
    \item .getScript().
\end{itemize}
\end{frame}

\subsection{Controladores Globales de Eventos} % (fold)

\begin{frame}[fragile]{Controladores Globales de Eventos} % (fold)
\begin{itemize}
    \item .ajaxComplete(handler) \textit{ handler= la función que se invoca.}
    \begin{lstlisting}
    $('.log').ajaxComplete(function() {
      $(this).log('Triggered ajaxComplete handler');
    });
    \end{lstlisting}
    \item .ajaxError().
    \item .ajaxStart().
    \item .ajaxStop().
    \item .ajaxSuccess().
    \item .ajaxSend(handler) \textit{ handler= la función que se invoca.} 
    \begin{lstlisting}
    $('.log').ajaxSend(function() {
      $(this).log('Triggered ajaxSend handler');
    });
    \end{lstlisting}
\end{itemize}
\end{frame}

\subsection{Funciones de Ayuda} % (fold)

\begin{frame}[fragile]{Funciones de Ayuda} % (fold)
\begin{itemize}
    \item .serialize(param)
    \begin{lstlisting}
    <form>
      <div><input type="text" name="a" value="1" id="a" /></div>
      <div><input type="text" name="b" value="2" id="b" /></div>
      <div><input type="text" name="c" value="3" id="c" /></div>
      <div><textarea name="d" rows="8" cols="40">4</textarea></div>
      <div><select name="e">
        <option value="5" selected="selected">5</option>
        <option value="6">6</option><option value="7">7</option>
      </select></div>
      <div><input type="checkbox" name="f" value="8" id="f" /></div>
      <div><input type="submit" name="g" value="Submit" id="g" /></div>
    </form>
    \end{lstlisting}
    \begin{lstlisting}
    $('form').submit(function() {
      $(this).log($('input, textarea, select').serialize());
      return false;
    });
    \end{lstlisting}
\end{itemize}
\end{frame}

\end{document}
