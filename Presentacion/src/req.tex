% Copyleft 2010 by Walter Vargas <walter@covetel.com.ve>

\documentclass[9pt]{beamer}
\usepackage{listings}
\lstset{
    language=HTML 
}
\lstset{
    basicstyle=\scriptsize, 
    stringstyle=\ttfamily,
    showstringspaces=false, 
    numbers=left,
    numberstyle=\scriptsize,
    tabsize=2,
}

% Configuración de la apariencia. 

\usetheme{Szeged}
\usecolortheme{beaver}
\usefonttheme[onlylarge]{structurebold}
\setbeamerfont*{frametitle}{size=\normalsize,series=\bfseries}

\usepackage{color}
\definecolor{lightgray}{rgb}{.9, .9, .9}
\definecolor{darkgray}{rgb}{.4, .4, .4}
\definecolor{purple}{rgb}{0.65,  0.12,  0.82}
\definecolor{azulito}{HTML}{0066CC}

\lstdefinelanguage{HTML}{
keywords={xml, DOCTYPE, head,  body, html, return,  this}, 
keywordstyle=\color{azulito}\bfseries, 
ndkeywords={title, option, form, textarea, table, th, tr, td, p, br, input,
button, frame, iframe, div, optgroup, select, fieldset,  legend, h1, h2, h3,
h4, h5, h6, p, blockquote, address, em,strong, abbr, acronym, cite, dfn, code,
kbd, samp, var, q, ul, li, dl, dt, dd, strong, span, add, attr, is, prev,
parent, children, parents, gt, find, lt, eq, not,  contains, filter, visible,
hidden, nth, even, odd, css}, 
ndkeywordstyle=\color{blue}\bfseries, 
identifierstyle=\color{black}, 
sensitive=false, 
comment=[l]{//}, 
morecomment=[s]{/*}{*/}, 
stringstyle=\color{azulito}\ttfamily, 
morestring=[b]', 
morestring=[b]"
}

\lstset{
language=HTML, 
backgroundcolor=\color{lightgray}, 
extendedchars=true, 
basicstyle=\scriptsize\ttfamily, 
showstringspaces=false, 
showspaces=false, 
numbers=left, 
numbersstyle=\footnotesize, 
numbersep=9pt, 
tabsize=2, 
breaklines=true, 
showtabs=false, 
captionpos=b 
}


% Paquetes
\usepackage[utf8]{inputenc}
\usepackage[spanish]{babel}


% Titulo
\title[WebDesign] {Diseño Web | Tecnologías Involucradas}
\author[Walter Vargas]{ info@covetel.com.ve \inst{1}}
\subtitle{Requerimientos Participantes del Curso}
\institute[covetel.com.ve]{ \inst{1} Cooperativa Venezolana de Tecnologías Libres R.S. }
\date

\begin{document}

\begin{frame} % (fold)
    \titlepage 
\end{frame}

\begin{frame} % (fold)
    \tableofcontents
\end{frame}

\section{Requerimientos Participantes del Curso} % (fold)

\subsection{Diferencia entre .attr() y .prop()} % (fold)

\begin{frame}[fragile]{Diferencia entre .attr y .prop()} % (fold)
En la versión 1.6 de JQuery se han dividido los métodos para el manejo de
atributos y propiedades DOM.
\begin{itemize}
\item Atributos DOM: \textit{representan el estado de la información DOM una vez recuperados del Documento.}

\textbf{Ejemplo}

\begin{lstlisting}
<input type="text" value="abc">
\end{lstlisting}

\item Propiedades DOM: \textit{representan el estado dinámico del documento.
.prop() selecciona u obtiene las propiedades de los elementos DOM.} 

\textbf{Ejemplo}

Si un usuario hace click en un elemento de entrada y define .prop("value") como abcdef .attr("value") seguirá siendo abc. 
\end{itemize}
\end{frame}

\begin{frame}[fragile]{Diferencia entre .attr y .prop()} % (fold)
En la mayoría de los casos,  el navegador trata el valor del atributo como el valor inicial,  pero los atributos booleanos como el checked o disabled tienen una semántica inusual.

\textbf{Ejemplo}

Considere la etiqueta:
\begin{lstlisting}
<input type="checkbox" checked>
\end{lstlisting}
La presencia del atributo checked significa que la propiedad DOM .checked es verdadera, a pesar que el atributo no tiene valor. En el código anterior el valor del atributo checked es una cadena vacía pero el valor de la propiedad checked es verdadero.

Antes de JQuery 1.6 .attr("checked") devolvía el valor de la propiedad booleana
(true) pero a partir de JQuery 1.6 devuelve el valor real del atributo (una
cadena vacía), que no cambia cuando el usuario hace click en la casilla de verificación para cambiar su estado.
\end{frame}

\begin{frame}[fragile]{Método .noConflict()} % (fold)
Muchas bibliotecas JavaScript usan \texttt{\$} función o nombre de variable,
como lo hace JQuery. \texttt{\$} es solo un alias para JQuery, por lo que toda
la funcionalidad está disponible sin necesidad de utilizar \texttt{\$}. Si
tenemos que usar otra librería de JavaScript junto con Jquery podemos devolver el control de \texttt{\$} a la biblioteca JavaScript usando el método
.noConflict():

\textbf{Ejemplo}

\begin{lstlisting}
 <script type="text/javascript" src="other_lib.js"> </ script>
 <script type="text/javascript" src="jquery.js"> </ script>
 <script type="text/javascript">
   . $ NoConflict ();
   jQuery (document). ready (function ($) {
   //El codigo que utiliza $ jQuery puede seguir aqui.
   });
   //El codigo que utiliza otras bibliotecas $ puede seguir aqui.
 </script>
\end{lstlisting}

\begin{lstlisting}
jQuery.noConflict();
//Algo con jQuery
jQuery("div p").hide();
//Algo con otra libreria $()
$("content").style.display = 'none';
\end{lstlisting}
\end{frame}

\begin{frame}[fragile]{Manejo de Objetos, Interacción de Array con JQuery} % (fold)
\begin{itemize}
\item .each() \textit{ : método de iteracción genérico, que puede ser usado para recorrer sin problemas los objetos y las matrices. Las matrices y los objetos de matriz, como una propiedad de longitud.}

\textbf{Ejemplo}

\begin{lstlisting}
<body>
  <div id="one"></div>
  <div id="two"></div>
  <div id="three"></div>
  <div id="four"></div>
  <div id="five"></div>
<script>
    var arr = [ "one", "two", "three", "four", "five" ];
    var obj = { one:1, two:2, three:3, four:4, five:5 };

    jQuery.each(arr, function() {
      $("#" + this).text("Mine is " + this + ".");
       return (this != "three"); // will stop running after "three"
   });

    jQuery.each(obj, function(i, val) {
      $("#" + i).append(document.createTextNode(" - " + val));
    });
</script>
</body>
\end{lstlisting}
\end{itemize}
\end{frame}

\begin{frame}[fragile]{Manejo de Objetos, Interacción de Array con JQuery} % (fold)
\begin{itemize}
\item .map() \textit{Traduce todos los elementos de una matriz o un objeto a
un nuevo array de items.}

\textbf{Ejemplo}

\begin{lstlisting}
<body>
  <div></div>
  <p></p>
  <span></span>
<script>
    var arr = [ "a", "b", "c", "d", "e" ];
    $("div").text(arr.join(", "));
    arr = jQuery.map(arr, function(n, i){
      return (n.toUpperCase() + i);
    });
    $("p").text(arr.join(", "));
    arr = jQuery.map(arr, function (a) { 
      return a + a; 
    });
    $("span").text(arr.join(", "));
</script>
</body>
\end{lstlisting}
\end{itemize}
\end{frame}

\begin{frame}[fragile]{Creación de Plugins JQuery} % (fold)
Para crear nuestro plugin, podemos extender de dos objetos:
\begin{itemize}
\item JQuery: que maneja la lógica interna de la librería.
\item JQuery.fn: que maneja interacciones con los elementos visuales
\end{itemize}
En Función del tipo de plugin extenderemos de uno u otro objeto. Nuestro plugin debemos incluirlo en una librería JavaScript que deberia llamarse: jquery.myplugin.js, donde myplugin es el nombre de nuestro plugin.

Creamos el objeto principal, sus metodos y propiedades

\textbf{Ejemplo}

\begin{lstlisting}
    jQuery.comprobarAlert = function(message){    
         alert(message);    
    }

    jQuery.fn.alerter = function(){
    this.each( function(){
        alert(this);
    });
    }
\end{lstlisting}
\end{frame}

\begin{frame}[fragile]{Creación de Plugins JQuery} % (fold)
Creamos un archivo HTML en el que vinculamos la librería JQuery y
nuestro plugin y lo usamos.

\begin{lstlisting}
$(document).ready( function(){
    $("#victima").alerter();
});
\end{lstlisting}

Podemos extender la opciones del plugin. Como ejemplo añadimos al plugin de
alerta la posibilidad de enviar una cadena adicional al comienzo y al final del mensaje, para esto hay dos opciones:
\begin{itemize}
\item La primera usar \texttt{JQuery.extend}: 'textit{ : una función de
built-in de JQuery, pensada para proveer variables "default" modificadas por el usuario.}
\begin{lstlisting}
jQuery.comprobarAlert = function( mensaje, opciones_user ){

    opciones_default = {
        inicio_str : "Alerta: ",
        final_str: " /fin "
    }


    opciones = jQuery.extend(opciones_default , opciones_user);
    alert(opciones.inicio_str + mensaje + opciones.final_str);
}
\end{lstlisting}
\end{itemize}
\end{frame}

\begin{frame}[fragile]{Creación de Plugins JQuery} % (fold)
\begin{itemize}
\item la segunda es hacerlo sin ayuda de extend que puede ser mas entendible.
\begin{lstlisting}
jQuery.comprobarAlert = function( mensaje, opciones_user ){
    if( opciones_user == undefined ){
        opciones = ({
            inicio_str : "Alerta: ",
            final_str: " /fin "
        });
    }else{
        opciones = opciones_user;
    }
    alert(opciones.inicio_str + mensaje + opciones.final_str);
}
\end{lstlisting}
\end{itemize}
\end{frame}

\begin{frame}[fragile]{Serialización de Arreglos y Objetos} % (fold)
\begin{itemize}
\item .param() \textit{ : crea una representación en serie de un array o un objeto, adecuado para usar en una cadena de consulta URL o solicitud Ajax.}
\begin{lstlisting}
var myObject = {
  a: {
    one: 1, 
    two: 2, 
    three: 3
  }, 
  b: [1,2,3]
};
var recursiveEncoded = $.param(myObject);
var recursiveDecoded = decodeURIComponent($.param(myObject));

alert(recursiveEncoded);
alert(recursiveDecoded);
\end{lstlisting}
\end{itemize}
\end{frame}

\begin{frame}[fragile]{Serialización de Arreglos y Objetos} % (fold)
\begin{itemize}
\item .serializeArray() \textit{Codifica un conjunto de elementos de formulario como una matriz de nombres y valores.}
\begin{lstlisting}
<form>
  <div><input type="text" name="a" value="1" id="a" /></div>
  <div><input type="text" name="b" value="2" id="b" /></div>
  <div><input type="hidden" name="c" value="3" id="c" /></div>
  <div>
    <textarea name="d" rows="8" cols="40">4</textarea>
  </div>
  <div><select name="e">
    <option value="5" selected="selected">5</option>
    <option value="6">6</option>
    <option value="7">7</option>
  </select></div>
  <div>
    <input type="checkbox" name="f" value="8" id="f" />
  </div>
  <div>
    <input type="submit" name="g" value="Submit" id="g" />
  </div>
</form>
\end{lstlisting}
\end{itemize}
\end{frame}

\begin{frame}[fragile]{Serialización de Arreglos y Objetos} % (fold)
\begin{lstlisting}
<script>
$('form').submit(function() {
  console.log($(this).serializeArray());
  return false;
});
</script>
\end{lstlisting}
Esto produce la siguiente estructura de datos (siempre que el navegador es compatible con console.log ):
\begin{lstlisting}
[
   {
     nombre: "a",
     valor "1"
   },
   {
     nombre: "b",
     valor: "2"
   }
]
\end{lstlisting}
\end{frame}

\begin{frame}[fragile]{Callback} % (fold)
Enlaza una función que se ejecuta cuando el DOM ha terminado de cargar.
\textbf{Ejemplo}
\begin{lstlisting}
 $(function(){ // Document is ready }); 
\end{lstlisting}
\begin{lstlisting}
 jQuery(function($) { // Your code using failsafe $ alias here... }); 
\end{lstlisting}
\end{frame}

\begin{frame}[fragile]{Llamadas Asíncronas} % (fold)
\begin{itemize}
\item jQuery.ajax() \textit{ : Realizar una solicitud asíncrona HTTP (Ajax).}
\item jQuery.ajax(url, [settings]) 
\begin{itemize}
\item Url \textit{ : una cadena que contiene la URL a la que se envía la solicitud.}
\item Settings \textit{ : un conjunto de pares llave/valor que configuran la
petición Ajax. Todas las configuraciones son opcionales. Un defecto se puede
configurar para cualquier opción con .ajaxSetup().}

\textit{Ver \texttt{http://api.jquery.com/jQuery.ajax/jQuery-ajax-settings} para ver una lista completa de todos los ajustes.}
\end{itemize}
Por defecto el método \texttt{.ajax()} realiza todas las solicitudes de forma asincrónica (es decir, se establece en true por defecto). Si necesita solicitudes sincrónicas, establezca esta opción false .  
\end{itemize}
\end{frame}

\begin{frame}[fragile]{Llamadas Asíncronas} % (fold)
\textbf{Ejemplo}
Recuperar la versión más reciente de una página HTML.
\begin{lstlisting}
$.ajax({
  url: "test.html",
  cache: false,
  success: function(html){
    $("#results").append(html);
  }
});
\end{lstlisting}
\end{frame}
\end{document}
