% Copyleft 2010 by Walter Vargas <walter@covetel.com.ve>

\documentclass[9pt]{beamer}
\usepackage{listings}
\lstset{
    language=HTML 
}
\lstset{
    basicstyle=\scriptsize, 
    stringstyle=\ttfamily,
    showstringspaces=false, 
    numbers=left,
    numberstyle=\scriptsize,
    tabsize=2,
}

% Configuración de la apariencia. 

\usetheme{Szeged}
\usecolortheme{beaver}
\usefonttheme[onlylarge]{structurebold}
\setbeamerfont*{frametitle}{size=\normalsize,series=\bfseries}

\usepackage{color}
\definecolor{lightgray}{rgb}{.9, .9, .9}
\definecolor{darkgray}{rgb}{.4, .4, .4}
\definecolor{purple}{rgb}{0.65,  0.12,  0.82}
\definecolor{azulito}{HTML}{0066CC}

\lstdefinelanguage{HTML}{
keywords={xml, DOCTYPE, head,  body, html, return,  this}, 
keywordstyle=\color{azulito}\bfseries, 
ndkeywords={title, option, form, textarea, table, th, tr, td, p, br, input,
button, frame, iframe, div, optgroup, select, fieldset,  legend, h1, h2, h3,
h4, h5, h6, p, blockquote, address, em,strong, abbr, acronym, cite, dfn, code,
kbd, samp, var, q, ul, li, dl, dt, dd, strong, span, add, attr, is, prev,
parent, children, parents, gt, find, lt, eq, not,  contains, filter, visible,
hidden, nth, even, odd, css}, 
ndkeywordstyle=\color{blue}\bfseries, 
identifierstyle=\color{black}, 
sensitive=false, 
comment=[l]{//}, 
morecomment=[s]{/*}{*/}, 
stringstyle=\color{azulito}\ttfamily, 
morestring=[b]', 
morestring=[b]"
}

\lstset{
language=HTML, 
backgroundcolor=\color{lightgray}, 
extendedchars=true, 
basicstyle=\scriptsize\ttfamily, 
showstringspaces=false, 
showspaces=false, 
numbers=left, 
numbersstyle=\footnotesize, 
numbersep=9pt, 
tabsize=2, 
breaklines=true, 
showtabs=false, 
captionpos=b 
}


% Paquetes
\usepackage[utf8]{inputenc}
\usepackage[spanish]{babel}


% Titulo
\title[WebDesign] {Diseño Web | Tecnologías Involucradas}
\author[Walter Vargas]{ info@covetel.com.ve \inst{1}}
\subtitle{Elementos de texto}
\institute[covetel.com.ve]{ \inst{1} Cooperativa Venezolana de Tecnologías Libres R.S. }
\date

\begin{document}

\begin{frame} % (fold)
    \titlepage 
\end{frame}

\begin{frame} % (fold)
    \tableofcontents
\end{frame}

\section{Elementos de texto}

\subsection{La creación de bloques de contenido} % (fold)

\begin{frame}{La creación de bloques de contenido} % (fold)
    Los elementos de texto se dividen en dos grandes categorías: en línea y en
    bloque. Los elementos en línea están en el flujo del texto y, por defecto,
    no causan saltos de línea. \\[0.5cm]
    \pause
    Los elementos a nivel de bloque, tienen una presentación por defecto que
    comienza en una nueva línea y tienden a apilarse como bloques en el flujo
    normal del documento. \\[0.5cm]
    \pause
    Los elementos de bloque componenen la estructura principal del
    documento.\\[0.5cm]
\end{frame}

\begin{frame}[fragile]{Cabeceras \texttt{<h1> <h6>}} % (fold)
    Las cabeceras se utilizan para introducir ideas o secciones de texto.
    (X)HTML define seis niveles de cabeceras, de \texttt{h1} a \texttt{h6}, de
    mayor a menor importancia. 

    \textbf{Atributos}
    \begin{itemize}
        \item Núcleo (\texttt{id, class, style, title}),
        \textit{Internacionalización},  \textit{Eventos}
    \end{itemize}

    \textbf{Atributos relegados}
    \begin{itemize}
        \item \texttt{align='center|left|right'}
    \end{itemize}

    Este ejemplo define el elemento como una cabecera de primer nivel

    \begin{lstlisting}
<h1> Perl Primeros Pasos </h1>
    \end{lstlisting}
\end{frame}

\begin{frame}[fragile]{Párrafos \texttt{<p> </p>}} % (fold)
    Los párrafos son los elementos más rudimentarios de un documento de texto.
    Se indican con el elemento \texttt{p}.\\[0.1cm]

    \textbf{Atributos}
    \begin{itemize}
        \item Núcleo (\texttt{id, class, style, title}),
        \textit{Internacionalización},  \textit{Eventos}
    \end{itemize}
    \textbf{Atributos relegados}
    \begin{itemize}
        \item \texttt{align='center|left|right'}
    \end{itemize}

    Los párrafos pueden contener texto y elementos en linea, pero no pueden
    contener otros elementos de bloque, como otros párrafos. Este es un ejemplo
    de un párrafo marcado como un elemento \texttt{p}

    \begin{lstlisting}
<p> Lorem ipsum dolor sit amet, consectetur adipisicing elit, sed do eiusmod tempor incididunt ut labore et dolore magna aliqua.
</p> 
    \end{lstlisting}

\end{frame}

\begin{frame}[fragile]{Citas \texttt{blockquote}} % (fold)
    Utilice el elemento \texttt{blockquote} para citas largas, especialmente
    aquellas que ocupan varios párrafos y tienen saltos de línea. \\[0.5cm]
    \textbf{Atributos}
    \begin{itemize}
        \item Núcleo (\texttt{id, class, style, title}),
        \textit{Internacionalización},  \textit{Eventos}
    \end{itemize}
    Se recomienda que el contenido de una cita esté contenido en otros
    elementos de bloque como párrafos, cabeceras, listas, etc, tal y como
    muestra este ejemplo de marcado. 

    \begin{lstlisting}
<p><blockquote
cite="http://catb.org/~esr/writings/taoup/html/ch01s06.html#id2877537"> 
Write simple parts connected by clean interfaces.
</blockquote> </p>
    \end{lstlisting}
\end{frame}

\subsubsection{Texto preformateado}

\begin{frame}{Texto preformateado \texttt{<pre> </pre>}} % (fold)
    El texto preformateado se utiliza cuando es necesario conservar el espacio
    en blanco de la fuente (espacios de caracteres o saltos de linea) al
    mostrar el documento en pantalla. Puede ser muy útil para código o poesía,
    donde los espacios y el alineamiento son importantes para el
    significado.\\[0.5cm]

    \textbf{Atributos}
    \begin{itemize}
        \item Núcleo (\texttt{id, class, style, title}),
        \textit{Internacionalización},  \textit{Eventos}
    \end{itemize}
    \textbf{Atributos depreciados} 
        \begin{itemize}
            \item \texttt{width='número'}
        \end{itemize}

     El texto preformateado es el único que aparece en pantalla exactamente
     como fue escrito en el código fuente. 
\end{frame}


\subsubsection{Direcciones}

\begin{frame}[fragile]{Direcciones \texttt{<address> ... </address>}} % (fold)
    El elemento \texttt{address} se utiliza para proporcionar información de
    contacto del autor o del encargado de mantenimiento del documento. No es
    apropiado para todos los listados de direcciones. 
    \textbf{Atributos}
    \begin{itemize}
        \item Núcleo (\texttt{id, class, style, title}),
        \textit{Internacionalización},  \textit{Eventos}
    \end{itemize}
    
    \begin{lstlisting}
<address>
Documento elaborado por Walter Vargas 
<a href="http://www.covetel.com.ve/"> Covetel R.S. </a> 
</address>
    \end{lstlisting}
\end{frame}


% subsection La creación de bloques de contenido (end)

\subsection{Elementos de línea} % (fold)

\begin{frame}{Elementos de Línea} % (fold)
    La mayoria de los elementos de texto son elementos de línea. Los elementos
    de línea, por defecto, no añaden saltos de línea ni expacios extra. 
\end{frame}

\subsubsection{Elementos de frase}

\begin{frame}{Elementos de frase} % (fold)

\footnotesize{
    HTML 4.01 y XHTML 1.0 y 1.1 definen una colección de elementos de frase
    (tambien denominados elementos lógicos) para añadir estructura y
    significado al texto en línea. Como los elementos de frase comparten
    sintaxis y atributos, se agregan a un elemento de los de aquí listados.\\[0.5cm]

    \textbf{abbr, acronym, cite, code, dfn, em, kbd, samp, strong, var}\\[0.5cm]

    \textbf{Atributos}
    \begin{itemize}
        \item Núcleo (\texttt{id, class, style, title}),
        \textit{Internacionalización},  \textit{Eventos}
    \end{itemize}

    Los elementos de frase pueden contener otros elementos en línea. En las
    próximas láminas se presenta el significado y uso de cada elemento. También
    aparecen indicados los elementos con una presentación estandarizada en los
    navegadores (por ejemplo los elementos \texttt{em} se muestran siempre en
    cursiva). Recuerde, sin embargo, seleccionar elementos en función del
    significado y no del efecto de renderizado. 
}
\end{frame}

\begin{frame}[fragile]{\texttt{em}} % (fold)
    Indica texto enfatizado, los elementos \texttt{em} casi siempre se
    renderizan en cursiva. 
    \begin{lstlisting}
<em>Emphasized text</em>
    \end{lstlisting}
\end{frame}

\begin{frame}[fragile]{\texttt{strong}} % (fold)
    Indica texto muy enfatizado, los elementos \texttt{strong} casi siempre se
    están renderizados en negrita.
    \begin{lstlisting}
<strong>Strong text</strong>
    \end{lstlisting}
\end{frame}

\begin{frame}[fragile]{\texttt{abbr}} % (fold)
    Indica una forma abreviada
    \begin{lstlisting}
The <abbr title="World Health Organization">WHO</abbr> was founded in 1948.
    \end{lstlisting}
\end{frame}

\begin{frame}[fragile]{\texttt{acronym}} % (fold)
    Indica un acrónimo.
\begin{lstlisting} 
this api allow connect to an <acronym title="Lightweight Directory Access Protocol">LDAP</acronym>
\end{lstlisting}
\end{frame}

\begin{frame}[fragile]{\texttt{cite}} % (fold)
    Indica una cita: una referencia a otro documento, especialmente libros,
    revistas, artículos, etc. Las citas suelen renderizarse en cursiva. 
\begin{lstlisting}
<cite>Especificacion XHTML 2</cite>
\end{lstlisting}
\end{frame}

\begin{frame}[fragile]{\texttt{dfn}} % (fold)
    Indica la definición del término. Puede utilizarser para llamar la
    atención sobre la introducción de términos y frases especiales. Estos
    términos suelen renderizarse en cursiva. 
    \begin{lstlisting}
<dfn> A hacker is ... </dfn>
    \end{lstlisting}
\end{frame}

\begin{frame}[fragile]{\texttt{code}} % (fold)
    Indica un ejemplo de código. Por defecto se renderiza en una fuente de tipo
    monoespacial especificada por el navegador (normalmente la tipografía
    Courier). 

\begin{lstlisting}
<code class="haskell">
fac :: Integer -> Integer
fac 0 = 1
fac n | n > 0 = n * fac (n-1)
</code>
\end{lstlisting}
\end{frame}

\begin{frame}[fragile]{kbd} % (fold)
    Equivale a \textbf{keyboard} o teclado e indica el texto introducido por el
    usuario. Puede ser útil para documentos técnicos. El texto de teclado suele
    renderizarse en fuente monoespacial. 

\begin{lstlisting}
Escriba en la consola <kbd> rm -f -R / </kbd> para ...
\end{lstlisting}
\end{frame}

\begin{frame}[fragile]{\texttt{samp}} % (fold)
    Indica muestras de salida de programas, \textit{scripts}, etc. Puede ser
    útil para documentos técnicos. El texto de muestra suele renderizarse en
    fuente monoespacial. 
\begin{lstlisting}
En la pantalla va a salir la frase 
<samp> Kernel panic </samp>
\end{lstlisting}
\end{frame}

\begin{frame}[fragile]{\texttt{var}} % (fold)
    Indica una variable o arguento de programa. Es otro elemento útil para
    documentos técnicos. Las variables suelen renderizarse en cursiva. 
\begin{lstlisting}
El elemento se encuentra en la variable 
<var> $_ </var>
\end{lstlisting}
\end{frame}

\begin{frame}[fragile]{Citas breves \texttt{q}} % (fold)
    HTML 4 introdujo el elemento \texttt{q} para indicar citas en línea breves
    como "Ser o no ser". Las citas más largas deberían utilizar el elemento
    \texttt{blockquote}. \\[0.5cm]

    \textbf{Atributos} 
    \begin{itemize}
        \item Núcleo (\texttt{id, class, style, title}),
        \textit{Internacionalización},  \textit{Eventos}
        \item \texttt{cite='url'}
    \end{itemize}

\begin{lstlisting}
<q cite="http://catb.org/~esr/writings/taoup/html/ch01s06.html"> 
Design for the future, because it will
be here sooner than you think
</q>
\end{lstlisting}
\end{frame}    

\begin{frame}[fragile]{Texto eliminado e insertado} % (fold)
\footnotesize{
    Los elementos \texttt{ins} y \texttt{del} se utilizan para marcar los
    cambios en el texto y para indicar partes de un documento que han sido
    insertadas o eliminadas. Pueden ser útiles en los documentos legales y en
    cualquier caso en el que se necesite seguir la pista a las
    modificaciones.\\[0.4cm]

    Pueden contener una o más palabras en un párrafo o uno o más elementos como
    párrafos, listas y tablas. \\[0.4cm]

    Cuando se utilizan como elementos en línea no pueden contener elementos de
    bloque porque violaría el contenido permitido.

    \textbf{Atributos}
    \begin{itemize}
        \item Núcleo (\texttt{id, class, style, title}),
        \textit{Internacionalización},  \textit{Eventos}
        \item \texttt{cite='url'}
        \item \texttt{datetime="YYYY-MM-DDThh:mm:ssTZD"}
    \end{itemize}

\begin{lstlisting}
Mi color favorito es <del title="explica breve del cambio">azul</del> <ins>rojo</ins>
\end{lstlisting}
 }
\end{frame}

\subsection{Los elementos genéricos \texttt{div} y \texttt{span} } % (fold)

\begin{frame}{Los elementos genéricos \texttt{div} y \texttt{span} } % (fold)
    Los elementos genéricos \texttt{div} y \texttt{span} permiten a los autores
    crear elementos personalizados. El elemento \texttt{div} se utiliza para
    indicar elementos a nivel de bloque y \texttt{span} indica elementos en
    línea. Estos dos elementos tienen atributos \texttt{id} y \texttt{class}
    para darles nombre, significado o contexto.
\end{frame}

\begin{frame}{El \texttt{div} polivalente} % (fold)
    El elemento \texttt{div} se utiliza para identificar y etiquetar cualquier
    división de texto a nivel de bloque, sean unos cuantos elementos de lista o
    una página entera. \\[0.4cm]

    \textbf{Atributos}
    \begin{itemize}
        \item Núcleo (\texttt{id, class, style, title}),
        \textit{Internacionalización},  \textit{Eventos}
    \end{itemize}

    \textbf{Atributos relegados}
    \begin{itemize}
        \item \texttt{align='center|left|right'}
    \end{itemize}
\end{frame}

\begin{frame}[fragile]{El útil \texttt{span}} % (fold)

\footnotesize{
    Como el elemento \texttt{div}, \texttt{span} permite a los autores crear
    elementos personalizados. La diferencia es que \texttt{span} se utiliza
    para elementos en línea y no introduce un salto de línea.\\[0.1cm]

    \textbf{Atributos}
    \begin{itemize}
        \item Núcleo (\texttt{id, class, style, title}),
        \textit{Internacionalización},  \textit{Eventos}
    \end{itemize}

    El código es un ejemplo de un \texttt{span} utilizado para identificar un
    número de teléfono. 

\begin{lstlisting}
Jenny: <span class="telephone">867.5309</span>
\end{lstlisting}
    
    Este marcado tiene distintos usos. Lo más común es utilizarlo como un
    "gancho" que puede aplicarse a reglas de hojas de estilo. En este ejemplo,
    todos los elementos etiquetados como \texttt{telephone} puede recibir las
    mismas instrucciones de presentación, como mostrarse en texto azul en
    negritas.\\[0.1cm]

    El \texttt{span} también da significado a una serie de otra forma aleatoria
    de dígitos para los agentes de usuario que conocen lo que hay que hacer con
    la información \texttt{telephone}. 

}
\end{frame}

\begin{frame}{Elementos identificadores \texttt{class}, \texttt{id}} % (fold)
    Los ejemplos anteriores muestran cómo se utilizan los atributos \texttt{id}
    y \texttt{class} para convertir elementos genéricos \texttt{div} y
    \texttt{span} en elementos con significados y usos específicos. \\[0.5cm]

    Debe destacarse que los atributos \texttt{class} y \texttt{id} pueden
    utilizarse con casi todos los elementos XHTML, no solo con \texttt{div} y
    \texttt{span}. 
\end{frame}

\begin{frame}{Identificador \texttt{id}} % (fold)
    El atributo \texttt{id} se utiliza para dar a un elemento un nombre
    específico y único en el documento. Esto significa que no habrá ningún otro
    elemento con ese \texttt{id}. Los valores \texttt{id} deben ser únicos. \\
    La recomendación HTML 4.01 especifica estos usos para el atributo
    \texttt{id}: 

    \begin{itemize}
        \item Como un selector de hojas de estilo
        \item Como ancla para vínculos 
        \item Como un medio para acceder a un elmento desde un \textit{script}
        \item Para el procesado por parte de agentes de usuario, esencialmente
        tratando los elementos como datos.
    \end{itemize}

\end{frame}

\begin{frame}{El identificador \texttt{class}} % (fold)
    El atributo \texttt{class} se utiliza para agrupar elementos similares.
    Pueden asignarse varios elementos al mismo nombre \texttt{class} y de este
    modo se pueden tratar igual. \\
    En el ejemplo anterior de \texttt{span}, el número de teléfono estaba
    identificado como \texttt{telephone} con el atributo \texttt{class}. Esto
    implica que puede haber muchos números de teléfono más en el documento. Una
    sola regla de hojas de estilo podría utilizarse para que todos sean en azul
    y negritas. \\
    De acuerdo a la especificación HTML 4.01 este atributo puede utilizarse: 

    \begin{itemize}
        \item Como selector de hojas de estilo
        \item Para el procesado por parte de agentes de usuario
    \end{itemize}
\end{frame}


\begin{frame}[fragile]{Notas sobre el uso de \texttt{class}} % (fold)
\footnotesize{
    La capacidad de crear sus propios elementos personalizados utlizando
    \texttt{id} y \texttt{class} es muy satisfactoria. El atributo
    \texttt{class} es propenso a ser mal utilizado. Estos consejos le darán
    algunas pautas básicas de marcado. 

    \begin{itemize}
        \item \textbf{Lograr que los nombres \texttt{class} tengan
        significado.} \\
        El valor del atributo \texttt{class} debería proporcionar una
        descripción semántica del contenido de un \texttt{div} o \texttt{span}.
        Elegir nombres en función de la presentación que se quiere dar al
        elemento no ayuda a dar significado al elemento y reintroduce la
        información de presentación al documento. 

        \item \textbf{No abuse de las clases: } Es fácil excederse asignando
        nombres \texttt{class} a elementos. En muchos casos pueden utilizarse
        otros tipos de selectores como selectores contextuales o de atributo.
        Por ejemplo, en lugar de etiquetar todos los elementos \texttt{h1} de
        una barra lateral como \texttt{class="barralateral"} podría utilizarse
        un selector contextual de este modo: 

        \begin{lstlisting}
div#sidebar h1 {font: Verdana 1.2em bold #444;}
        \end{lstlisting}
    \end{itemize}
    }
\end{frame}

\subsection{Listas} % (fold)

\begin{frame}{Listas} % (fold)
    Los humanos somos creadores de listas por naturaleza, así que resulta
    lógico que los mecanismos para crear listas de información hayan formado
    parte del lenguaje HTML desde el principio. Esta sección presenta los tipos
    de lista definidos en XHTML. 
    \begin{itemize}
        \item Información sin numerar
        \item Información numerada
        \item Términos y definiciones
    \end{itemize}
\end{frame}


\begin{frame}{Listas no numeradas \texttt{<ul> <li> </li> </ul>}} % (fold)
    Las listas no numeradas se utilizan para colecciones de elementos
    relacionados que aparecen sin ningún orden determinado. \\[0.5cm]

    \textbf{Elemento \texttt{<ul>}}\\[0.2cm] 

    \textbf{Atributos}
    \begin{itemize}
        \item Núcleo (\texttt{id, class, style, title}),
        \textit{Internacionalización},  \textit{Eventos}
    \end{itemize}

    \textbf{Atributos relegados}
    \begin{itemize}
        \item \texttt{compact} 
        \item \texttt{type="disc|circle|square"}
    \end{itemize}
\end{frame}


\begin{frame}{Listas no numeradas \texttt{<ul> <li> </li> </ul>}} % (fold)
    \textbf{Elemento \texttt{<li>}}\\[0.2cm] 

    \textbf{Atributos}
    \begin{itemize}
        \item Núcleo (\texttt{id, class, style, title}),
        \textit{Internacionalización},  \textit{Eventos}
    \end{itemize}

    \textbf{Atributos relegados}
    \begin{itemize}
        \item \texttt{type='formato'} 
        \item \texttt{value='número'}
    \end{itemize}
\end{frame}

\begin{frame}[fragile]{Sintaxis de listas no numeradas} % (fold)
    Este ejemplo muestra el marcado de una lista básica no ordenada. 

    \begin{lstlisting}
<ul> 
    <li> Perl  </li>
    <li> Haskel </li> 
    <li> Erlang </li>
    <li> Lisp </li> 
</ul>
    \end{lstlisting}

    En HTML 4.01 las etiquetas de cierre para los elementos de lista son
    opcionales, pero en XHTML se necesitan todas las etiquetas de cierre. Es
    aconsejable cerrar todos los elementos sea cual sea la versión de HTML que
    se esta utilizando.
\end{frame}

\begin{frame}{Listas numeradas \texttt{<ol>}} % (fold)
    Las listas numeradas se utilizan para las listas en las que la secuencia de
    los elementos es importante, como instrucciones paso a paso o notas al
    final de página. Las listas numeradas se indican con el elemento
    \texttt{ol} y deben incluir uno o más elementos de lista \texttt{li}. Como
    todas las listas, las listas numeradas y sus elementos de lista son
    elementos a nivel de bloque. 

    \textbf{Atributos}
    \begin{itemize}
        \item Núcleo (\texttt{id, class, style, title}),
        \textit{Internacionalización},  \textit{Eventos}
    \end{itemize}

    \textbf{Atributos relegados}
    \begin{itemize}
        \item \texttt{compact}
        \item \texttt{start='number'}
        \item \texttt{type='1|A|a|I|i'}
    \end{itemize}
\end{frame}

\begin{frame}[fragile]{Sintaxis de listas numeradas} % (fold)
    Las listas numeradas tienen las mismas estructuras básicas que las listas no
    numeradas como muestra este ejemplo. 

    \begin{lstlisting}
<ol> 
    <li> Despertarse </li> 
    <li> Cepillarse </li>
    <li> Tomar una ducha </li> 
    <li> Preparar Te </li> 
    <li> Revisar el correo </li> 
</ol>
    \end{lstlisting}

\end{frame}

\begin{frame}{Listas de definición \texttt{<dl>}} % (fold)
   Utilice una lista de definición para las listas que concisten en parejas de
   términos y definiciones. \\[0.2cm]
    
   \textbf{Atributos del elemento \texttt{dl}}
   \begin{itemize}
       \item Núcleo (\texttt{id, class, style, title}),
       \textit{Internacionalización},  \textit{Eventos}
       \item \texttt{compact}
   \end{itemize}
   
   \textbf{Atributos del elemento \texttt{dd}}
   \begin{itemize}
       \item Núcleo (\texttt{id, class, style, title}),
       \textit{Internacionalización},  \textit{Eventos}
   \end{itemize}
\end{frame}


\begin{frame}[fragile]{Ejemplo de listas de definición} % (fold)
\begin{lstlisting}
<dl>
  <dt>Coffee</dt>
    <dd>- black hot drink</dd>
  <dt>Milk</dt>
    <dd>- white cold drink</dd>
</dl>
\end{lstlisting}
\end{frame}

% subsection Listas (end)

\end{document}
