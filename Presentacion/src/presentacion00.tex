% Copyleft 2010 by Walter Vargas <walter@covetel.com.ve>

\documentclass{beamer}
\usepackage{listings}
\lstset{language=HTML,CSS,XHTML,XML}

% Configuración de la apariencia. 

\usetheme{Szeged}
\usecolortheme{beaver}
\usefonttheme[onlylarge]{structurebold}
\setbeamerfont*{frametitle}{size=\normalsize,series=\bfseries}

% Paquetes
\usepackage[utf8]{inputenc}
\usepackage[spanish]{babel}

% Titulo
\title[WebDesign] {Diseño Web | Tecnologías Involucradas}
\author[Walter Vargas]{ info@covetel.com.ve \inst{1}}
\institute[covetel.com.ve]{ \inst{1} Cooperativa Venezolana de Tecnologías Libres R.S. }
\date

\begin{document}

\begin{frame}
  \titlepage
\end{frame}

\section{El Entorno Web}
\begin{frame}{Estándares Web} % (fold)

\begin{center}
Hoy en día, hablar de diseño Web es hablar de adecuación a los estándares.
\pause \\[0.5cm]
Hoy en día, hablar de estándares es hablar de la W3C.
\end{center}
\end{frame}


\begin{frame}{¿Qué es la W3C?} % (fold)
\begin{center}
El {\bfseries World Wide Web Consortium}, abreviado W3C, es un consorcio
internacional creado por {\bfseries Tim Berners-Lee}\footnote{Tim Berners-Lee
es el creador de el lenguaje HTML, el protocolo HTTP y  el sistema URL} que produce recomendaciones para la World Wide Web.
\\[0.5cm]
\pause

\includegraphics{imgs/w3c.png}
\end{center}

\end{frame}

\begin{frame}{¿Qué son los estándares?} % (fold)

    \begin{center}
    La W3C crea y supervisa el desarrollo de tecnologías Web, incluyendo XML,
    HTML y sus distintas aplicaciones. \\[0.5cm]
    \pause

    También controla problemas de más alto nivel como la accesibilidad de los
    contenidos al mayor número posible de dispositivos y usuarios.\\[0.5cm]
    \pause
    El W3C no es un cuerpo oficial de estándares, es un esfuerzo en común por
    parte de los expertos en materias relacionadas con la Web por poner
    {\bfseries orden} en el desarrollo de las tecnologías Web.
    \end{center}
\end{frame}


\begin{frame}{Otros Cuerpos de Estándares que afectan la Web} % (fold)
    \begin{itemize}
        \item ISO (International Organization for Standardization.)
        \item IETF (Internet Engineering Task Force) 
        \item Ecma International
        \item Unicode Consortium
        \item ANSI
    \end{itemize}
\end{frame}

\begin{frame}{Las ventajas de los estándares} % (fold)
    \begin{center}
    Se espera que los navegadores se adherirán llenos de fe y fidelidad a las
    recomendaciones de la W3C. 
    \end{center}
    \begin{itemize}
        \item Accesibilidad
        \item Comptatibilidad proactiva
        \item Desarrollo más rápido y sencillo 
        \item Descarga y muestra en pantallas más rápidas. 
    \end{itemize}
\end{frame}

\begin{frame}{Herramientas para desarrollar apegados a los estándares} % (fold)
    \begin{center}
    Existen herramientas que nos ayudan a escribir nuestras aplicaciones más
    apegadas a los estándares.
    \end{center}
\end{frame}

\section{Herramientas}

\begin{frame}{Firefox} % (fold)
    \begin{center}
    \begin{itemize}
        \item Web Developer \footnote{\url{https://addons.mozilla.org/en-US/firefox/addon/web-developer/}}
        \item HTML Validator \footnote{\url{http://users.skynet.be/mgueury/mozilla/}}
    \end{itemize}
    \end{center}
\end{frame}

\begin{frame}{W3C} % (fold)
    \begin{center}
    \begin{itemize}
        \item Instalar el validador local de la W3C para HTML/XHTML
        \footnote{aptitude install w3c-markup-validator} 
        \item Instalar el validador local de la W3C para CSS \footnote{\url{http://jigsaw.w3.org/css-validator/DOWNLOAD.html}}
    \end{itemize}
    \end{center}
\end{frame}

\begin{frame}[fragile]{Instalar el validador de HTML} % (fold)
    \begin{center}
    Podemos instalar el validador de la W3C para HTML en un sistema Debian
    Lenny/Squeeze utilizando el siguiente comando:
    \begin{verbatim}
    # aptitude install w3c-markup-validador
    \end{verbatim}
    \end{center}
\end{frame}

\begin{frame}[fragile]{Instalar las dependencias del validador de CSS} % (fold)
    Podemos instalar el validador de la W3C para CSS en un sistema Debian
    Lenny/Squeeze utilizando los siguientes comandos:
    {\footnotesize
    \begin{verbatim}
    # aptitude install sun-java6-jdk 
    # aptitude install ant 
    # aptitude install tomcat5.5 tomcat5.5-admin tomcat5.5-webapps
    # aptitude install cvs 
    \end{verbatim}
    }
\end{frame} 

\begin{frame}[fragile]{Instalar el validador de CSS} % (fold)

    {\footnotesize 
    \begin{verbatim}
    # cd /usr/src 
    # export CVSROOT=:pserver:anonymous@dev.w3.org:/sources/public 
    # cvs login 
      password: anonymous
    # export CVSROOT=:pserver:anonymous@dev.w3.org:/sources/public 
    # cvs checkout 2002/css-validator
    # cd 2002/css-validator 
    # wget http://www.covetel.com.ve/conf/build.xml
    # ant war 
    # cp css-validator.war /var/lib/tomcat5.5/webapps 
    \end{verbatim}
    }
    TOMCAT5\_SECURITY=NO en el archivo /etc/default/tomcat5.5\\[0.5cm]

    \begin{verbatim}
    # /etc/init.d/tomcat5.5 restart
    \end{verbatim}

\end{frame}

\begin{frame}{Preguntas} % (fold)
    \begin{center}
    Siempre puede enviar sus preguntas y hacer seguimiento del curso a la
    dirección de correo {\bfseries cursos@covetel.com.ve}
    \end{center}
\end{frame}

\end{document}
