% Copyleft 2010 by Walter Vargas <walter@covetel.com.ve>

\documentclass{beamer}
\usepackage{listings}
\lstset{
    language=HTML 
}
\lstset{
    basicstyle=\scriptsize, 
    stringstyle=\ttfamily,
    showstringspaces=false
}

% Configuración de la apariencia. 

\usetheme{Szeged}
\usecolortheme{beaver}
\usefonttheme[onlylarge]{structurebold}
\setbeamerfont*{frametitle}{size=\normalsize,series=\bfseries}

\usepackage{listings}
\usepackage{color}
\definecolor{lightgray}{rgb}{.9, .9, .9}
\definecolor{darkgray}{rgb}{.4, .4, .4}
\definecolor{purple}{rgb}{0.65,  0.12,  0.82}
\definecolor{azulito}{HTML}{0066CC}

\lstdefinelanguage{HTML}{
keywords={xml, DOCTYPE, head,  body, html}, 
keywordstyle=\color{green}\bfseries, 
ndkeywords={title, table, th, tr, td, p, br, input,  button, frame, iframe, div}, 
ndkeywordstyle=\color{blue}\bfseries, 
identifierstyle=\color{black}, 
sensitive=false, 
comment=[l]{//}, 
morecomment=[s]{/*}{*/}, 
stringstyle=\color{azulito}\ttfamily, 
morestring=[b]', 
morestring=[b]"
}

\lstset{
language=HTML, 
backgroundcolor=\color{lightgray}, 
extendedchars=true, 
basicstyle=\footnotesize\ttfamily, 
showstringspaces=false, 
showspaces=false, 
numbers=left, 
numberstyle=\footnotesize, 
numbersep=9pt, 
tabsize=2, 
breaklines=true, 
showtabs=false, 
captionpos=b 
}

% Paquetes
\usepackage[utf8]{inputenc}
\usepackage[spanish]{babel}

% Titulo
\title[WebDesign] {Diseño Web | Tecnologías Involucradas}
\author[Walter Vargas]{ info@covetel.com.ve \inst{1}}
\institute[covetel.com.ve]{ \inst{1} Cooperativa Venezolana de Tecnologías Libres R.S. }
\date

\begin{document}

\begin{frame}{Tabla de Contenidos} % (fold)
\tableofcontents
\end{frame}

\begin{frame}
  \titlepage
\end{frame}

\section{El Entorno Web}
\begin{frame}{Estándares Web} % (fold)

\begin{center}
Hoy en día, hablar de diseño Web es hablar de adecuación a los estándares.
\pause \\[0.5cm]
Hoy en día, hablar de estándares es hablar de la W3C.
\end{center}
\end{frame}


\begin{frame}{¿Qué es la W3C?} % (fold)
\begin{center}
El {\bfseries World Wide Web Consortium}, abreviado W3C, es un consorcio
internacional creado por {\bfseries Tim Berners-Lee}\footnote{Tim Berners-Lee
es el creador de el lenguaje HTML, el protocolo HTTP y  el sistema URL} que produce recomendaciones para la World Wide Web.
\\[0.5cm]
\pause

\includegraphics{imgs/w3c.png}
\end{center}

\end{frame}

\begin{frame}{¿Qué son los estándares?} % (fold)

    \begin{center}
    La W3C crea y supervisa el desarrollo de tecnologías Web, incluyendo XML,
    HTML y sus distintas aplicaciones. \\[0.5cm]
    \pause

    También controla problemas de más alto nivel como la accesibilidad de los
    contenidos al mayor número posible de dispositivos y usuarios.\\[0.5cm]
    \pause
    El W3C no es un cuerpo oficial de estándares, es un esfuerzo en común por
    parte de los expertos en materias relacionadas con la Web por poner
    {\bfseries orden} en el desarrollo de las tecnologías Web.
    \end{center}
\end{frame}


\begin{frame}{Otros Cuerpos de Estándares que afectan la Web} % (fold)
    \begin{itemize}
        \item ISO (International Organization for Standardization.)
        \item IETF (Internet Engineering Task Force) 
        \item Ecma International
        \item Unicode Consortium
        \item ANSI
    \end{itemize}
\end{frame}

\begin{frame}{Las ventajas de los estándares} % (fold)
    \begin{center}
    Se espera que los navegadores se adherirán llenos de fe y fidelidad a las
    recomendaciones de la W3C. 
    \end{center}
    \begin{itemize}
        \item Accesibilidad
        \item Comptatibilidad proactiva
        \item Desarrollo más rápido y sencillo 
        \item Descarga y muestra en pantallas más rápidas. 
    \end{itemize}
\end{frame}

\begin{frame}{Herramientas para desarrollar apegados a los estándares} % (fold)
    \begin{center}
    Existen herramientas que nos ayudan a escribir nuestras aplicaciones más
    apegadas a los estándares.
    \end{center}
\end{frame}

\subsection{Herramientas}

\begin{frame}{Firefox} % (fold)
    \begin{center}
    \begin{itemize}
        \item Web Developer \footnote{\url{https://addons.mozilla.org/en-US/firefox/addon/web-developer/}}
        \item HTML Validator \footnote{\url{http://users.skynet.be/mgueury/mozilla/}}
    \end{itemize}
    \end{center}
\end{frame}

\begin{frame}{W3C} % (fold)
    \begin{center}
    \begin{itemize}
        \item Instalar el validador local de la W3C para HTML/XHTML
        \footnote{aptitude install w3c-markup-validator} 
        \item Instalar el validador local de la W3C para CSS \footnote{\url{http://jigsaw.w3.org/css-validator/DOWNLOAD.html}}
    \end{itemize}
    \end{center}
\end{frame}

\begin{frame}[fragile]{Instalar el validador de HTML} % (fold)
    \begin{center}
    Podemos instalar el validador de la W3C para HTML en un sistema Debian
    Lenny/Squeeze utilizando el siguiente comando:
    \begin{verbatim}
    # aptitude install w3c-markup-validador
    \end{verbatim}
    \end{center}
\end{frame}

\begin{frame}[fragile]{Instalar las dependencias del validador de CSS} % (fold)
    Podemos instalar el validador de la W3C para CSS en un sistema Debian
    Lenny/Squeeze utilizando los siguientes comandos:
    {\footnotesize
    \begin{verbatim}
    # aptitude install sun-java6-jdk 
    # aptitude install ant 
    # aptitude install tomcat5.5 tomcat5.5-admin tomcat5.5-webapps
    # aptitude install cvs 
    \end{verbatim}
    }
\end{frame} 

\begin{frame}[fragile]{Instalar el validador de CSS} % (fold)

    {\footnotesize 
    \begin{verbatim}
    # cd /usr/src 
    # export CVSROOT=:pserver:anonymous@dev.w3.org:/sources/public 
    # cvs login 
      password: anonymous
    # export CVSROOT=:pserver:anonymous@dev.w3.org:/sources/public 
    # cvs checkout 2002/css-validator
    # cd 2002/css-validator 
    # wget http://www.covetel.com.ve/conf/build.xml
    # ant war 
    # cp css-validator.war /var/lib/tomcat5.5/webapps 
    \end{verbatim}
    }
    TOMCAT5\_SECURITY=NO en el archivo /etc/default/tomcat5.5\\[0.5cm]

    \begin{verbatim}
    # /etc/init.d/tomcat5.5 restart
    \end{verbatim}

\end{frame}

\begin{frame}{Preguntas} % (fold)
    \begin{center}
    Siempre puede enviar sus preguntas y hacer seguimiento del curso a la
    dirección de correo {\bfseries cursos@covetel.com.ve}
    \end{center}
\end{frame}

\subsection{Estándares Web actuales}

\begin{frame}{Estándares Web actuales} % (fold)
    \begin{center}
        Esta sección presenta los estándares actuales para los siguientes
        aspectos: 
        \pause
        \begin{itemize}
            \item Estructura
            \item Comportamiento
            \item Presentación del diseño Web
        \end{itemize}
    \end{center}
\end{frame}

\begin{frame}{Modelo de Capas del Diseño Web} % (fold)
    \begin{center}
        Suele hablarse del diseño y el desarrollo Web en términos de
        \textbf{capas}.
        \begin{itemize}
            \item El marcado de documentos conforma la {\bfseries Capa
            Estructural}
            \item La {\bfseries Capa de Presentación} especificada con las
            hojas de estilo en cascada (CSS)
            \item La {\bfseries Capa de Comportamiento} contiene el
            \textit{scripting} y programación que añade interactividad y
            efectos dinámicos a un sitio. 
        \end{itemize}
    \end{center}

\end{frame}

\subsubsection{Capa estructural}

\begin{frame}{Capa estructural} % (fold)
\begin{center}
    Lenguajes estándar actuales para el marcado estructural
    \begin{itemize}
        \item XHTML 1.0 y XHTML 1.1 \footnote{XHTML son las siglas de
        \textit{Extensible Hypertext Markup Language} o Lenguaje Extensible de
        Marcado de Hipertexto}
        \item XML 1.0 \footnote{XML son las siglas de \textit{Extensible Markup
        Language}} 
    \end{itemize}
\end{center}
\end{frame}

\begin{frame}{XHTML} % (fold)
    \begin{center}
        XHTML 1.0 es simplemente HTML 4.0 reescrito bajo las mas estrictas
        reglas sintácticas de XML. \\[0.5cm]
        \pause
        XHTML 1.1 se libera de elementos y atributos depreciados y se ha
        modularizado para facilitar expansiones futuras.\\[0.5cm]
        \pause
        XHTML Aún esta en desarrollo. (XHTML2)\\[0.5cm]
        \pause
        La última versión de HTML fue HTML 4.01, sigue siendo soportado por los
        navegadores actuales, pero no será compatible a largo plazo.\\[0.5cm]
        \pause
        Puede encontrar los vinculos de estos lenguajes en la dirección
        \url{http://www.w3c.org/MarkUp}
    \end{center}
\end{frame}

\begin{frame}{XML} % (fold)
    \begin{center}
       XML es un grupo de reglas para la creación de nuevos lenguajes de
       marcado.  \\[0.5cm]

       Esto les permite los desarrolladores crear grupos personalizados de
       etiquetas para usos especiales. \\[0.5cm]

       Para encontrar más información valla a la fuente en
       \url{http://www.w3c.org/XML}
    \end{center}
\end{frame}

\subsubsection{Capa de Presentación} 

\begin{frame}{Capa de Presentación} % (fold)
    \begin{center}
        Ahora que todas las instrucciones de presentación fueron eliminadas del
        estándar de marcado,  esta información pasa a ser un trabajo exclusivo
        de las hojas de estilo en cascada.\\[0.5cm]

        Los estándares de las hojas de estilo se están desarrollando en fases
        de este modo: 

        \begin{itemize}
            \item \textbf{CSS Nivel 1}
            \item \textbf{CSS Nivel 2}
            \item \textbf{CSS Nivel 3}
        \end{itemize}

   
    \end{center}
\end{frame}

\begin{frame}{CSS Nivel 1} % (fold)
    \begin{center}
        Este estándar de hojas de estilo ha sido una recomendación desde el año
        1996 y ahora tienen soporte completo por las versiones actuales de los
        navegadores. \\[0.5cm]

        Este nivel contiene reglas que controlan la presentación en pantalla
        del texto, márgenes y bordes. 
    \end{center}
\end{frame}

\begin{frame}{CSS Nivel 2.1} % (fold)
    \begin{center}
    Esta recomendación es conocida por la adición de la característica de
    posicionamiento absoluto de los elementos de la página Web. El nivel 2
    alcanzó el estatus de recomendación en 1998. \\[0.5cm]

    En 2004 el nivel 2.1 fue aceptada como recomendación oficial 

    Puede encontrar una traducción al español de la recomendación 2.1 en la
    dirección \url{http://www.w3.org/Style/css21-updates/css2.1_spa.pdf}
    \end{center}
\end{frame}

\begin{frame}{CSS Nivel 3} % (fold)
    El nivel 3 se construye sobre el nivel 2, pero esta modularizado para
    facilitar las expansiones futuras y para permitir a los dispositivos
    soportar conjuntos lógicos.\\[0.5cm] 

    La actualización del nivel 3 incluye nuevas características gráficas como
    los bordes redondeados, textos con sombras, asignar multiples fondos, un
    mejor manejo de las tablas incluyendo el estilo zebra, multi-columnas. El
    modelo conservará muchas de las propiedades actuales y trabajará con nuevos
    selectores. 
\end{frame}

\subsubsection{Capa de Comportamiento} 

\begin{frame}{Capa de Comportamiento} % (fold)
\begin{center}
    
   El \textit{scripting} y programación de esta capa añaden interactividad y
   efectos dinámicos a un sitio. \\[0.5cm]

   La capa de comportamiento descanza sobre \textbf{DOM} (\textit{Document Object
   Model}, Modelo de Objetos del Documento).  
\end{center}
\end{frame}

\begin{frame}{Modelos de objeto} % (fold)
\begin{center}
    El DOM permite a scripts y aplicaciones acceder y modificar el contenido,
    la estructura y el estilo de un documento nombrando formalmente cada una de
    sus partes, sus atributos y el modo en el que el objeto puede ser
    manipulado.  \\[0.5cm]

    Al principio cada uno de los navegadores tenia su propio DOM, dificultando
    la creación de efectos interactivos para todos los navegadores. 
\end{center}
\end{frame}

\begin{frame}{Niveles de DOM} % (fold)
    \begin{center}
        \begin{itemize}
            \item Modelo de Objetos del Documento (Núcleo) Nivel 1
            \item Modelo de Objetos del Documento Nivel 2
        \end{itemize}
    \end{center}
\end{frame}

\begin{frame}{DOM 1} % (fold)
    \begin{center}
        DOM Nivel 1 incluye documentos XML y HTML así como la manipulación y
        navegación de documentos. Esta especificación puede encontrarse en la
        dirección \url{http://www.w3c.org/TR/REC-DOM-Level-1}
    \end{center}
\end{frame}

\begin{frame}{DOM 2} % (fold)
    \begin{center}
        DOM Nivel 2 incluye un modelo de objetos de hojas de estlo, permitiendo
        manipular información de estilo. En la dirección
        \url{http://www.w3c.org/DOM/DOMTR} puede encontrar más información.
    \end{center}
\end{frame}

\subsubsection{Scripting} 



\begin{frame}{Scripting} % (fold)
    \begin{center}
     Netscape introdujo su lenguaje de programación Web,  JavaScript, con su
     navegador Navigator 2.0. Originalmente se llamaba \textit{Livescript} pero
     cuando la marca pasó a ser compartida por SUN, se añadió \textit{Java} al nombre.\\[0.5cm]

     Microsoft respondió con su propio \textit{JScript}, mientras soportaba
     algunos niveles de JavaScript en la versión 3.0 de su navegador. \\[0.5cm]

     Se necesitaba claramente un estándar válido para los distintos
     navegadores.
    \end{center}
\end{frame}

\begin{frame}{JavaScript 1.5/ECMAScript 262} % (fold)
    \begin{center}
        La W3C desarrolla una versión estándarizada de JavaScript en
        coordinación con Ecma International.\\[0.5cm]

        En la mayoría de los casos los desarrolladores se refieren al lenguaje
        como JavaScript, siendo implícita la implementación estándar.\\[0.5cm]

        Puede encontrar la especificación completa en la dirección
        \url{http://www.ecma-international.org/publications/standards/Ecma-262.htm}
    \end{center}
\end{frame}

\section{Estructura del documento} % (fold)
\label{sec:Estructura del documento}

\begin{frame}{Estructura del documento} % (fold)
    Un documento (X)HTML se compone de tres partes: 

    \begin{itemize}
        \item Una \textbf{declaración} de la vesión HTML o XHTML utilizada. 
        \item Una \textbf{cabecera} con información sobre el documento.
        \item El  \textbf{cuerpo} con el contenido del documento.
    \end{itemize}
\end{frame}

\begin{frame}{Elementos utilizados para establecer la estructura del documento} % (fold)
    \begin{center}
        \begin{itemize}
            \item \texttt{html} Elemento raíz de un documento (X)HTML
            \item \texttt{head} Cabecera
            \item \texttt{body} Cuerpo del documento
            \item \texttt{title} Título del documento
            \item \texttt{meta} Metadatos (información sobre el documento)
        \end{itemize}
    \end{center}
\end{frame}

\begin{frame}[fragile]{Estructura mínima de un documento} % (fold)
    \begin{lstlisting}    
<?xml version="1.0" encoding="UTF-8"?>
<!DOCTYPE html PUBLIC "-//W3C//DTD XHTML 1.0 Strict//EN"
    "http://www.w3c.org/TR/xhtml1-strict.dtd">

<html xmlns="http://www.w3c.org/1999/xhtml" 
    xml:lang="es" lang="es"> 
    <head>
        <title> Titulo del documento </title>
    </head>

    <body>
        <p> Contenido del documento </p>
    </body>
</html>
    \end{lstlisting}
\end{frame}

\begin{frame}[fragile]{Declaración del tipo de documento DOCTYPE} % (fold)
    \begin{center}
       \begin{lstlisting}
 <!DOCTYPE html PUBLIC "-//W3C//DTD XHTML 1.0 Strict//EN"
    "http://www.w3c.org/TR/xhtml1-strict.dtd">
       \end{lstlisting}

       La declaración \texttt{<!DOCTYPE>} (tipo de documento) contiene dos
       métodos para señalar la información del DTD, un identificador de tipo de
       documento reconocido públicamente, o una url específica en caso de que
       el identificador público no sea reconocido por el dispositivo
    \end{center}
\end{frame}

\begin{frame}{Opciones DTD HTML 4.01 y XHTML 1.0} % (fold)
    \begin{center}
        \begin{itemize}
            \item Estricto
            \item Transicional
            \item Con marcos
        \end{itemize}
    \end{center}
\end{frame}

\begin{frame}{Lista de las descripciones y marcados} % (fold)
    \begin{center}
        \begin{itemize}
            \item HTML 4.01 estricto
            \item HTML 4.01 transicional
            \item HTML 4.01 con marcos 
            \item XHTML 1.0 estricto
            \item XHTML 1.0 transicional 
            \item XHTML 1.0 con marcos
            \item XHTML 1.1 
        \end{itemize}
    \end{center}
\end{frame}

\lstset{language=HTML}
\begin{frame}[fragile]{HTML 4.01 estricto} % (fold)
    \begin{center}
        La DTD estricta omite todos los elementos y atributos depreciados.
    \end{center}
    \begin{lstlisting}
<!DOCTYPE HTML PUBLIC "-//W3C//DTD HTML 4.01//EN"
   "http://www.w3.org/TR/html4/strict.dtd">
    \end{lstlisting} 
\end{frame}

\begin{frame}[fragile]{HTML 4.01 transicional} % (fold)
    \begin{center}
        La DTD transicional incluye toda la DTD estricta junto con todos los
        elementos depreciados. 
    \end{center}
    \begin{lstlisting}
<!DOCTYPE HTML PUBLIC "-//W3C//DTD HTML 4.01 Transitional//EN"
   "http://www.w3.org/TR/html4/loose.dtd">
    \end{lstlisting} 
\end{frame}

\begin{frame}[fragile]{HTML 4.01 con marcos} % (fold)
    \begin{center}
        Si el documento contiene marcos (es decir si utiliza \texttt{frameset}
        en lugar de \texttt{body}) identifique la DTD con marcos. Esta DTD es
        la misma que la versión transicional (incluye elementos y atributos
        depreciados) con elementos específicos para marcos. Los documentos HTML
        que contienen contenido que se muestran en los marcos no necesitan
        utilizar esta DTD.
    \end{center}
    \begin{lstlisting}
<!DOCTYPE HTML PUBLIC "-//W3C//DTD HTML 4.01 Frameset//EN"
   "http://www.w3.org/TR/html4/frameset.dtd">
    \end{lstlisting} 
\end{frame}

\begin{frame}[fragile]{XHTML 1.0 estricto} % (fold)
    \begin{center}
        Es igual al HTML 4.01 estricto pero reformulado de acuerdo a las reglas
        sintácticas del XML
    \end{center}
    \begin{lstlisting}
<!DOCTYPE html PUBLIC "-//W3C//DTD XHTML 1.0 Strict//EN"
   "http://www.w3.org/TR/xhtml1/DTD/xhtml1-strict.dtd">
    \end{lstlisting} 
\end{frame}

\begin{frame}[fragile]{XHTML 1.0 transicional} % (fold)
    \begin{center}
        Igual al HTML 4.01 transicional pero reformulado de acerdo a las reglas
        sitácticas del XML
    \end{center}
    \begin{lstlisting}
<!DOCTYPE html PUBLIC "-//W3C//DTD XHTML 1.0 Transitional//EN"
   "http://www.w3.org/TR/xhtml1/DTD/xhtml1-transitional.dtd">
    \end{lstlisting} 
\end{frame}

\begin{frame}[fragile]{XHTML 1.0 con marcos} % (fold)
    \begin{center}
        Igual que HTML 4.01 con marcos pero reformulado de acuerdo a las reglas
        sintácticas del XML
    \end{center}
    \begin{lstlisting}
<!DOCTYPE html PUBLIC "-//W3C//DTD XHTML 1.0 Frameset//EN"
   "http://www.w3.org/TR/xhtml1/DTD/xhtml1-frameset.dtd">
    \end{lstlisting} 
\end{frame}


\begin{frame}[fragile]{XHTML 1.1} % (fold)
    \begin{center}
        Solo hay una DTD para XHTML 1.1. Omite todos los elementos y atributos
        depreciados. Se distingue del XHTML 1.0 estricto en lo siguiente: 

        \begin{itemize}
            \item El atributo \texttt{lang} ha sido reemplazado por el atributo
            \texttt{xml:lang}
            \item El atributo \texttt{name} ha sido reemplazado por el atributo
            \texttt{id} para los elementos \texttt{a} y \texttt{map}
            \item Se ha añadido una colección \textit{ruby} de elementos. La
            W3C define \textit{ruby} como \textit{comandos cortos de texto al
            lado del texto base, típicamente utilizados en documentos
            asiáticos para indicar pronunciación o para proveer una nota breve}
        \end{itemize}
    \end{center}
    \begin{lstlisting}
<!DOCTYPE html PUBLIC "-//W3C//DTD XHTML 1.1//EN" 
   "http://www.w3.org/TR/xhtml11/DTD/xhtml11.dtd">
    \end{lstlisting} 
\end{frame}

\begin{frame}[fragile]{Elemento raíz} % (fold)
    \begin{center}
        \begin{lstlisting}
<html xmlns="http://www.w3c.org/1999/xhtml" 
    xml:lang="es" lang="es"> 
        \end{lstlisting}
    \end{center}
\end{frame}

\begin{frame}[fragile]{Cabecera del documento} % (fold)
    \begin{center}
        \begin{lstlisting}
<head>
  <title> Titulo del documento </title>
</head>
        \end{lstlisting}
    \end{center}
\end{frame}

\begin{frame}[fragile]{Cuerpo del documento} % (fold)
    \begin{center}
        \begin{lstlisting}
<body>
   <p> Contenido del documento </p>
</body>
        \end{lstlisting}
    \end{center}
\end{frame}


\begin{frame}{Preguntas} % (fold)
    \begin{center}
        Preguntas
    \end{center}
\end{frame}

\end{document}
