% Copyleft 2010 by Walter Vargas <walter@covetel.com.ve>

\documentclass{beamer}
\usepackage{listings}
\lstset{
    language=HTML 
}
\lstset{
    basicstyle=\scriptsize, 
    stringstyle=\ttfamily,
    showstringspaces=false, 
    numbers=left,
    numberstyle=\scriptsize,
    tabsize=2,
}

% Configuración de la apariencia. 

\usetheme{Szeged}
\usecolortheme{beaver}
\usefonttheme[onlylarge]{structurebold}
\setbeamerfont*{frametitle}{size=\normalsize,series=\bfseries}

% Paquetes
\usepackage[utf8]{inputenc}
\usepackage[spanish]{babel}

% Titulo
\title[WebDesign] {Diseño Web | Tecnologías Involucradas}
\author[Walter Vargas]{ info@covetel.com.ve \inst{1}}
\subtitle{Tablas y Formularios}
\institute[covetel.com.ve]{ \inst{1} Cooperativa Venezolana de Tecnologías Libres R.S. }
\date

\begin{document}

\begin{frame}{Tabla de Contenidos} % (fold)
\tableofcontents
\end{frame}

\begin{frame}
  \titlepage
\end{frame}

\section{Tablas}

\begin{frame}{Tablas} % (fold)
\begin{center}
    Los elementos HTML de tablas fueron introducidos por Netscape en la versión
    1.1 de su navegador, fueron desarrollados para permitir a los autores
    representar datos tabulares en filas y columnas.
\end{center}
\end{frame}

\subsection{Elementos de una Tabla} % (fold)
\label{sub:Elementos de una Tabla}

% subsection Elementos de una Tabla (end)

\begin{frame}{Elementos de una Tabla} % (fold)
    \begin{center}
        \begin{itemize}
            \item \texttt{table} Establecer una tabla
            \item \texttt{tr} Fila
            \item \texttt{td} Columna
            \item \texttt{th} Celda de cabecera de tabla 
            \item \texttt{caption} Leyenda de la tabla
            \item \texttt{thead} Cabecera de la tabla
            \item \texttt{tbody} Cuerpo de la tabla 
            \item \texttt{tfoot} Pie de una tabla 
            \item \texttt{col} Declara una columna
            \item \texttt{colgroup} Declara un grupo de columnas 
           
        \end{itemize}
    \end{center}
\end{frame}

\begin{frame}{Tipos de Tablas} % (fold)
   \begin{center}
       \begin{itemize}
           \item Tablas de Datos
           \item Tablas de Diseño \footnote{La recomendación HTML 4.01 de la
           W3C desaconseja específicamente este uso de tablas}
       \end{itemize}
   \end{center}
\end{frame}

\subsection{Estructura básica de las Tablas} % (fold)

\begin{frame}{Filas y Celdas} % (fold)
\begin{center}
    Los elementos mínimos para definir una tabla son \texttt{table},
    \texttt{tr}, \texttt{td}
\end{center}
\end{frame}

\begin{frame}{Atributos de la Tabla} % (fold)
    \begin{center}
        {\footnotesize
        \begin{itemize}
            \item \texttt{id, class, style, title} Núcleo,
            \textit{Internacionalización,  Eventos}
            \item \texttt{border="número"}
            \item \texttt{cellpadding="número"}
            \item \texttt{cellspacing="número"}
            \item \texttt{frame="void|above|below|hsides|lhs|rhs|vsides|box|border"}
            \item \texttt{summary}
            \item \texttt{width="número o porcentaje"}
        \end{itemize}
        }
    \end{center}
\end{frame}

\begin{frame}{Atributos Depreciados} % (fold)
    \begin{center}
        \begin{itemize}
            \item \texttt{align="left|right|center"}
            \item \texttt{bgcolor='color'}
        \end{itemize}
    \end{center}
\end{frame}

\begin{frame}{Atributos no estándar} % (fold)
    \begin{center}
        \begin{itemize}
            \item \texttt{height="número, porcentaje"}
        \end{itemize}
    \end{center}
\end{frame}

\begin{frame}{Filas \texttt{<tr>}} % (fold)
    \texttt{<tr>...</tr>}\\[0.5cm]
    \pause
    \textbf{Atributos}
    \begin{itemize}
        \item Núcleo (\texttt{id, class, style, title}),
        \textit{Internacionalización}, Eventos
        \item \texttt{aligh="left|center|right|justify|char"}
        \item \texttt{char='caracter'}
        \item \texttt{charoff="longitud"}
        \item \texttt{valign="top|middle|bottom|baseline"}
    \end{itemize}
    \pause 
    \textbf{Atributos relegados}
    \begin{itemize}
        \item \texttt{bgcolor='color'}
    \end{itemize}
\end{frame}

\begin{frame}{Celdas \texttt{<td>}} % (fold)
        \texttt{<td>...</td>}\\
        \textbf{Atributos}
        {\footnotesize 
        \begin{itemize}
            \item \textit{Núcleo} (\texttt{id, class, style, title}), \textit{Internacionalización}, \textit{Eventos}
            \item \texttt{abbr="texto"}
            \item \texttt{align="left|right|center|justify|char"}
            \item \texttt{axis="texto"}
            \item \texttt{char='character'}
            \item \texttt{charoff="longitud"}
        \end{itemize}
        }
\end{frame}

\begin{frame}{Celdas \texttt{<td>}} % (fold)
    \texttt{<td>...</td>}\\
    \textbf{Atributos}
    {\footnotesize 
    \begin{itemize}
            \item \texttt{colspan="número"}
            \item \texttt{headers="referencias"}
            \item \texttt{rowspan="número"}
            \item \texttt{scope="row|col|rowgroup|colgroup"}
            \item \texttt{valign="top|middle|bottom|baseline"}
    \end{itemize}
    }
\end{frame}

\begin{frame}{Celdas \texttt{<td>}} % (fold)
        \texttt{<td>...</td>}\\
        \textbf{Atributos relegados}
        \begin{itemize}
            \item \texttt{bgcolor}
            \item \texttt{height}
            \item \texttt{nowrap}
            \item \texttt{width}
        \end{itemize}
\end{frame}
% subsection Estructura básica de las Tablas (end)

\subsection{Expandir celdas y columnas} % (fold)
\label{sub:Expandir celdas y columnas}

\begin{frame}[fragile]{Expandir columnas \texttt{<colspan>} } % (fold)
        Para expandir columnas, utilizamos la etiqueta \texttt{colspan}. Por
        ejemplo: 

        \begin{lstlisting}
<table width="100%" border="1">
  <tr>
    <th>Month</th>
    <th>Savings</th>
  </tr>
  <tr>
    <td colspan="2">January</td>
  </tr>
  <tr>
    <td colspan="2">February</td>
  </tr>
</table>            
        \end{lstlisting}
\end{frame}

\begin{frame}[fragile]{Expandir filas \texttt{<rowspan>}} % (fold)
    El atributo \texttt{rowspan} estira una celda para que pase a ocupar el
    espacio de varias celdas, esta vez en sentido vertical. 
    \begin{lstlisting}
<table width="100%" border="1">
  <tr>
    <th>Month</th>
    <th>Savings</th>
    <th>Savings for holiday!</th>
  </tr>
  <tr>
    <td>January</td>
    <td>$100</td>
    <td rowspan="2">$50</td>
  </tr>
  <tr>
    <td>February</td>
    <td>$80</td>
  </tr>
</table>
    \end{lstlisting}
\end{frame}

% subsection Expandir celdas y columnas (end)

\subsection{Elementos descriptivos} % (fold)

\begin{frame}[fragile]{Cabeceras de tabla \texttt{<th>}} % (fold)
    Las celdas que forman la cabecera de la tabla se indican utilizando el
    elemento  \texttt{<th>} \\
    \begin{lstlisting}
<table border="1">
  <tr>
    <th>Month</th>
    <th>Savings</th>
  </tr>
  <tr>
    <td>January</td>
    <td>$100</td>
  </tr>
</table>
    \end{lstlisting}
    El elemento \texttt{<th>} acepta la misma lista de atributos que el
    elemento \texttt{<td>}
\end{frame}

\begin{frame}[fragile]{Leyendas \texttt{<caption>} } % (fold)
    El elemento \texttt{caption} da un título o una descripción breve de la
    tabla\\[0.2cm]
    \textbf{Atributos}
    \begin{itemize}
        \item \texttt{id, class, style, title} 
    \end{itemize}


    \textbf{Atributos relegados}
    \begin{itemize}
        \item \texttt{align="top|bottom|left|right"}
    \end{itemize}

\end{frame}

\begin{frame}[fragile]{Leyendas \texttt{<caption>} } % (fold)
\begin{lstlisting}[]
<table border="1">
  <caption>Monthly savings</caption>
  <tr>
    <th>Month</th>
    <th>Savings</th>
  </tr>
  <tr>
    <td>January</td>
    <td>$100</td>
  </tr>
</table>    
\end{lstlisting}
\end{frame}

% subsection Elementos descriptivos (end)

\subsection{Grupos de filas y columnas} % (fold)
\label{sub:Grupos de filas y columnas}

\begin{frame}{\texttt{<thead>, <tbody>, <tfoot>}} % (fold)
Grupos de filas, fueron introducidos por Internet Explorer 3.0 para que los
agentes de usuarios y las hojas de estilo pudieran tratarlas como unidades 

\begin{itemize}
    \item \textit{Núcleo} (\texttt{id, class, style, title}),
    \textit{Internacionalización, Eventos}
    \item \texttt{align="left|center|right|justify|char"}
    \item \texttt{char='caracter'}
    \item \texttt{charoff="longitud"}
    \item \texttt{valign="top|middle|bottom|baseline"}
\end{itemize}
    
\end{frame}

\begin{frame}[fragile]{Ejemplo \texttt{<thead>, <tbody>, <tfoot>}} % (fold)
\begin{lstlisting}
<table border="1">
  <thead>
    <tr> <th>Month</th> <th>Savings</th> </tr>
  </thead>
  <tfoot>
    <tr> <td>Sum</td> <td>$180</td> </tr>
  </tfoot>
  <tbody>
    <tr> <td>January</td> <td>$100</td> </tr>
    <tr> <td>February</td> <td>$80</td> </tr>
  </tbody>
</table>
\end{lstlisting}
\end{frame}

\begin{frame}{Columnas y grupos de columnas} % (fold)
\begin{center}
    En algunos casos es útil identificar columnas conceptuales de celdas de
    datos o grupos de columnas. Los elementos \texttt{col} (columna) y
    \texttt{colgroup} (grupo de columnas) permiten unir conceptualmente un
    grupo de celdas que aparezcan en una columna o columnas.\\[0.2cm]
    \pause
    Cuando se combinan estos grupos con el elemento \texttt{scope} puede dar un
    contexto muy útil a los lectores de pantallas. \\[0.2cm]
    \pause
    Los grupos de columnas permiten aplicar las siguientes propiedades de
    estilo de acuerdo a la recomendación CSS 2.1: \texttt{border, background,
    width, visibility}
\end{center}
\end{frame}

\begin{frame}{Columnas \texttt{<col />}} % (fold)
    \textbf{Atributos}
    \begin{itemize}
        \item Núcleo (\texttt{id, class, style, title}),
        \textit{Internacionalización, Eventos}
         \item \texttt{align="left|right|center|justify|char"}
         \item \texttt{span="número"}
         \item \texttt{width="pixels, porcetaje, n*"}
        \item \texttt{char='caracter'}
        \item \texttt{charoff="longitud"}
        \item \texttt{valign="top|middle|bottom|baseline"}
    \end{itemize}
    
\end{frame}

\begin{frame}[fragile]{Ejemplo de columnas \texttt{<col />}} % (fold)
\begin{lstlisting}
<table width="100%" border="1">
  <col width="20px" /> 
  <col width="30px" />
  <col width="10px" />
  <tr> 
    <th>ISBN</th> <th>Title</th> <th>Price</th>
  </tr>
  <tr>
    <td>3476896</td><td>My first HTML</td><td>$53</td>
  </tr>
  <tr>
    <td>2489604</td><td>My first CSS</td><td>$47</td>
  </tr>
</table>
\end{lstlisting}
\end{frame}

\begin{frame}[fragile]{Grupos de Columnas \texttt{<colgroup>...</colgroup>}} % (fold)
    Posee los mismos atributos que el elemento \texttt{<col />}. 
    \begin{lstlisting}
<table width="100%" border="1">
  <colgroup span="2" align="left"></colgroup>
  <colgroup align="right" style="color:#0000FF;"></colgroup>
  <tr>
    <th>ISBN</th>
    <th>Title</th>
    <th>Price</th>
  </tr>
  <tr>
    <td>3476896</td>
    <td>My first HTML</td>
    <td>$53</td>
  </tr>
</table>
        
    \end{lstlisting}
\end{frame}
% subsection Grupos de filas (end)

\subsection{Espaciado} % (fold)
\label{sub:Espaciado}

\begin{frame}{Espaciado de celdas} % (fold)
    Hay dos tipos de espacio que puede añadirse en y alrededor de las celdas:
    relleno de celda o \textit{cellpadding}, esto es el margen interior o el
    espacio entre una celda y su contenido y el espacio entre celdas o
    \textit{cellspacing}. \\[0.5cm]

    Los atributos \texttt{cellspacing} y \texttt{celpadding} se utilizan con el
    elemento \texttt{table}

\end{frame}

% subsection Espaciado (end)

\subsection{Cabeceras de tabla} % (fold)
\label{sub:Cabeceras de tabla}

\begin{frame}[fragile]{Asociar cabeceras con datos} % (fold)
    {\tiny
    Los atributos \texttt{scope} y \texttt{headers} se pueden utilizar para
    vincular los datos a las cabeceras. 
        }
\begin{lstlisting}
<table border="1">
  <tr>
    <th></th>
    <th scope="col">Month</th>
    <th scope="col">Savings</th>
  </tr>
  <tr>
    <td scope="row">1</td>
    <td>January</td>
    <td>$100</td>
  </tr>
  <tr>
    <td scope="row">2</td>
    <td>February</td>
    <td>$80</td>
  </tr>
</table>    
\end{lstlisting}
\end{frame}


\begin{frame}{Preguntas} % (fold)
    \begin{center}
    Preguntas
    \end{center}
\end{frame}


% subsection Cabeceras de tabla (end)

\end{document}
