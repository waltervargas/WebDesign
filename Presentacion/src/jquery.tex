\input{preambulo.tex}

% Titulo
\title[WebDesign] {Diseño Web | Tecnologías Involucradas}
\author[Walter Vargas]{ info@covetel.com.ve \inst{1}}
\subtitle{JQuery}
\institute[covetel.com.ve]{ \inst{1} Cooperativa Venezolana de Tecnologías Libres R.S. }
\date

\begin{document}

\begin{frame} % (fold)
    \titlepage 
\end{frame}

\begin{frame} % (fold)
    \tableofcontents
\end{frame}

\section{Anatomía de un Script de JQuery}

\subsection{Tablas Dinámicas de Contenido} % (fold)

\begin{frame}{JQuery} % (fold)
    JQuery es una biblioteca que contiene una amplia variedad de métodos y
    funciones de los cuales sera muy útil saber las categorías
    básicas y como utilizarlas para nuestro beneficio.\\[0.5cm]
\end{frame}

\begin{frame}[fragile]{Obtención de JQuery} % (fold)
    El sitio web oficial de JQuery siempre es el recurso más al día para el
    código y noticias relacionadas con la biblioteca. Para empezar necesitamos
    una copia de JQuery, que se puede descargar desde la página principal del
    sitio. Varias versiones de JQuery pueden estar disponibles en cualquier
    momento dado, la última versión sin comprimir será la más apropiada para
    nosotros.\\[0.5cm]
    Para usar JQuery no es necesaria su instalación sólo tenemos que colocarlo
    en nuestro sitio en un lugar público. Cada vez que necesitamos un metodo de
    JQuery simplemente lo referenciamos desde el archivo del documento
    HTML.\\[0.5cm]
\end{frame}

\begin{frame}[fragile]{Configuración del documento HTML} % (fold)
Existen tres componentes en el uso de JQuery:

\textbf{Componentes}
\begin{itemize}
    \item El Documento HTML.
    \item El Archivo CSS con los estilos.
    \item Los Archivos Javascript.
\end{itemize}

En este ejemplo se muestra como referenciar los componentes de JQuery.\\[0.5cm]

\end{frame}

\begin{frame}[fragile]{Configuración del documento HTML} % (fold)
\begin{lstlisting}
<html xmlns="http://www.w3.org/1999/xhtml" xml:lang:"en" lang:"en">
 <head>
  <meta http-equiv="Content-Type" content="text/html; charset=utf-8"/>
  <title>JQuery</title>
  <link rel="stylesheet" href="jquery.css" type="text/css" />
  <script src="jquery.js" type="text/javascript"></script>
  <script src="appjquery.js" type="text/javascript"></script>
  </head>
 <body>
 </body>
</html>
\end{lstlisting}

\end{frame}

\begin{frame}[fragile]{Escribiendo el Código JQuery} % (fold)
Nuestro código personalizado lo hemos incluido en el código HTML utilizando
<script src="appjquery" type="text/javascript"></script>. A pesar de lo mucho
que realiza tiene pocas lineas de código:
\end{frame}

\begin{frame}[fragile]{Escribiendo el Código JQuery} % (fold)
\begin{lstlisting}
JQuery.fn.toggleNext = function() {
 this.toggleClass('arrow-down')
 .next().slideToggle('fast');
}

$(document).ready(function () {
 $('div id="page-contents"></div>')
 .prepend('<h3>Page Contents</h3>')
 .append('<div></div>')
 .prependTo('body');
\end{lstlisting}
\end{frame}

\begin{frame}[fragile]{Escribiendo el Código JQuery} % (fold)
\begin{lstlisting}[firstnumber=last]
 $('#content h2').each(function(index) {
 var $chapterTitle = $(this);
 var chapterId = 'chapter~' + (index + 1);
 chapterTitle.attr('id', chapterId);
 $('<a></a>').text($chapterTitle.text())
  .attr({
   'title': 'Jump to' + $chapterTitle.text(), 
   'href': '#' + chapterId
  })
  .appendTo('#page-contents div');
  });
 \end{lstlisting}
\end{frame}

\begin{frame}[fragile]{Escribiendo el Código JQuery} % (fold)
 \begin{lstlisting}[firstnumber=last]
 $('#page-contents h3').click(function() {
  $(this).toggleNext();
 });

 $('#introduction > h2 a').click(function() {
  $('#introduction').load(this.href);
  return false;
 });
});
\end{lstlisting}
Este ejemplo crea una tabla dinámica para los usuarios.
\end{frame}
\end{document}
