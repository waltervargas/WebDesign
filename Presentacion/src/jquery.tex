% Copyleft 2010 by Walter Vargas <walter@covetel.com.ve>

\documentclass[9pt]{beamer}
\usepackage{listings}
\lstset{
    language=HTML 
}
\lstset{
    basicstyle=\scriptsize, 
    stringstyle=\ttfamily,
    showstringspaces=false, 
    numbers=left,
    numberstyle=\scriptsize,
    tabsize=2,
}

% Configuración de la apariencia. 

\usetheme{Szeged}
\usecolortheme{beaver}
\usefonttheme[onlylarge]{structurebold}
\setbeamerfont*{frametitle}{size=\normalsize,series=\bfseries}

\usepackage{color}
\definecolor{lightgray}{rgb}{.9, .9, .9}
\definecolor{darkgray}{rgb}{.4, .4, .4}
\definecolor{purple}{rgb}{0.65,  0.12,  0.82}
\definecolor{azulito}{HTML}{0066CC}

\lstdefinelanguage{HTML}{
keywords={xml, DOCTYPE, head,  body, html, return,  this}, 
keywordstyle=\color{azulito}\bfseries, 
ndkeywords={title, option, form, textarea, table, th, tr, td, p, br, input,
button, frame, iframe, div, optgroup, select, fieldset,  legend, h1, h2, h3,
h4, h5, h6, p, blockquote, address, em,strong, abbr, acronym, cite, dfn, code,
kbd, samp, var, q, ul, li, dl, dt, dd, strong, span, add, attr, is, prev,
parent, children, parents, gt, find, lt, eq, not,  contains, filter, visible,
hidden, nth, even, odd, css}, 
ndkeywordstyle=\color{blue}\bfseries, 
identifierstyle=\color{black}, 
sensitive=false, 
comment=[l]{//}, 
morecomment=[s]{/*}{*/}, 
stringstyle=\color{azulito}\ttfamily, 
morestring=[b]', 
morestring=[b]"
}

\lstset{
language=HTML, 
backgroundcolor=\color{lightgray}, 
extendedchars=true, 
basicstyle=\scriptsize\ttfamily, 
showstringspaces=false, 
showspaces=false, 
numbers=left, 
numbersstyle=\footnotesize, 
numbersep=9pt, 
tabsize=2, 
breaklines=true, 
showtabs=false, 
captionpos=b 
}


% Paquetes
\usepackage[utf8]{inputenc}
\usepackage[spanish]{babel}


% Titulo
\title[WebDesign] {Diseño Web | Tecnologías Involucradas}
\author[Walter Vargas]{ info@covetel.com.ve \inst{1}}
\subtitle{JQuery}
\institute[covetel.com.ve]{ \inst{1} Cooperativa Venezolana de Tecnologías Libres R.S. }
\date

\begin{document}

\begin{frame} % (fold)
    \titlepage 
\end{frame}

\begin{frame} % (fold)
    \tableofcontents
\end{frame}

\section{Anatomía de un Script de JQuery}

\subsection{Tablas Dinámicas de Contenido} % (fold)

\begin{frame}{JQuery} % (fold)
    JQuery es una biblioteca que contiene una amplia variedad de métodos y
    funciones de los cuales sera muy útil saber las categorías
    básicas y como utilizarlas para nuestro beneficio.\\[0.5cm]
\end{frame}

\begin{frame}[fragile]{Obtención de JQuery} % (fold)
    El sitio web oficial de JQuery siempre es el recurso más al día para el
    código y noticias relacionadas con la biblioteca. Para empezar necesitamos
    una copia de JQuery, que se puede descargar desde la página principal del
    sitio. Varias versiones de JQuery pueden estar disponibles en cualquier
    momento dado, la última versión sin comprimir será la más apropiada para
    nosotros.\\[0.5cm]
    Para usar JQuery no es necesaria su instalación sólo tenemos que colocarlo
    en nuestro sitio en un lugar público. Cada vez que necesitamos un metodo de
    JQuery simplemente lo referenciamos desde el archivo del documento
    HTML.\\[0.5cm]
\end{frame}

\begin{frame}[fragile]{Configuración del documento HTML} % (fold)
Existen tres componentes en el uso de JQuery:

\textbf{Componentes}
\begin{itemize}
    \item El Documento HTML.
    \item El Archivo CSS con los estilos.
    \item Los Archivos Javascript.
\end{itemize}

En este ejemplo se muestra como referenciar los componentes de JQuery.\\[0.5cm]

\end{frame}

\begin{frame}[fragile]{Configuración del documento HTML} % (fold)
\begin{lstlisting}
<html xmlns="http://www.w3.org/1999/xhtml" xml:lang:"en" lang:"en">
 <head>
  <meta http-equiv="Content-Type" content="text/html; charset=utf-8"/>
  <title>JQuery</title>
  <link rel="stylesheet" href="jquery.css" type="text/css" />
  <script src="jquery.js" type="text/javascript"></script>
  <script src="appjquery.js" type="text/javascript"></script>
  </head>
 <body>
 </body>
</html>
\end{lstlisting}

\end{frame}

\begin{frame}[fragile]{Escribiendo el Código JQuery} % (fold)
Nuestro código personalizado lo hemos incluido en el código HTML utilizando
\texttt{<script src="appjquery.js" type="text/javascript"></script>}. A pesar de lo mucho
que realiza tiene pocas lineas de código:
\end{frame}

\begin{frame}[fragile]{Escribiendo el Código JQuery} % (fold)
\begin{lstlisting}
JQuery.fn.toggleNext = function() {
 this.toggleClass('arrow-down')
 .next().slideToggle('fast');
}

$(document).ready(function () {
 $('div id="page-contents"></div>')
 .prepend('<h3>Page Contents</h3>')
 .append('<div></div>')
 .prependTo('body');
\end{lstlisting}
\end{frame}

\begin{frame}[fragile]{Escribiendo el Código JQuery} % (fold)
\begin{lstlisting}[firstnumber=last]
 $('#content h2').each(function(index) {
 var $chapterTitle = $(this);
 var chapterId = 'chapter~' + (index + 1);
 chapterTitle.attr('id', chapterId);
 $('<a></a>').text($chapterTitle.text())
  .attr({
   'title': 'Jump to' + $chapterTitle.text(), 
   'href': '#' + chapterId
  })
  .appendTo('#page-contents div');
  });
 \end{lstlisting}
\end{frame}

\begin{frame}[fragile]{Escribiendo el Código JQuery} % (fold)
 \begin{lstlisting}[firstnumber=last]
 $('#page-contents h3').click(function() {
  $(this).toggleNext();
 });

 $('#introduction > h2 a').click(function() {
  $('#introduction').load(this.href);
  return false;
 });
});
\end{lstlisting}
Este ejemplo crea una tabla dinámica para los usuarios.
\end{frame}

\subsection{Disección del Script} % (fold)

\begin{frame}[fragile]{Disección del Script} % (fold)
Este script ha sido elegido específicamente ya que ilustra la capacidad general
de la biblioteca JQuery e identifica las categorías de los métodos utilizados.
\end{frame}

\begin{frame}[fragile]{Selector de Expresiones} % (fold)
    Antes de que podamos actuar en un Documento HTML se debe localizar las
    partes pertinentes,  en el script algunas veces utilizamos un método simple
    para buscar un elemento:
\begin{lstlisting}
$('#introduction')
\end{lstlisting}
    Esta expresión se crea un nuevo objeto JQuery que hace referencia al
    elemento de ID \texttt{introduction}, algunas veces se requiere un selector
    más complejo:
\begin{lstlisting}
$('#introduction > h2 a')
\end{lstlisting}
   Aqui se obtiene un objeto JQuery que se refiere a los elementos
   descendientes de \texttt{<h2>} y que son hijos del elemento con ID
   \texttt{introduction}.   
\end{frame}

\begin{frame}[fragile]{Métodos de recorrido DOM} % (fold)
    A veces tenemos un objeto JQuery que ya hace referencia a un conjunto de
    elementos DOM,  pero tenemos que realizar una acción en uno diferente,  en
    estos casos los métodos de recorrido DOM son útiles, esto lo podemos ver en
    parte de nuestro script:
\begin{lstlisting}
this.toggleClass('arrow-down')
 .next()
 .slideToggle('fast');
\end{lstlisting}
    Debido al contexto de esta pieza de código, la palabra reservada
    \texttt{this} se refiere al objeto JQuery en este caso el objeto apunta la
    etiqueta \texttt{<h3>}. La llamada al método \texttt{.toggleClass} manipula
    este elemento de partida,  el método \texttt{.next()} cambia el
    funcionamiento del elemento y
    \texttt{.slideToggle} llama las acciones del elemento \texttt{<div>}
\end{frame}

\begin{frame}[fragile]{Métodos de Eventos} % (fold)
    Aun cuando podemos modificar la página a nuestra voluntad necesitamos
    métodos que reaccionen a eventos causados por los usuarios en el momento
    adecuado:
\begin{lstlisting}
$('#introduction > h2 a').click(function() {
 $('#introduction').load(this.href);
 return false;
});
\end{lstlisting}
    En este fragmento se registra un controlador que se ejecutará cada vez que
    se hace click en la etiqueta de anclaje seleccionado.
\end{frame}

\begin{frame}[fragile]{Métodos de Eventos} % (fold)
\begin{lstlisting}
$(document).ready(function() {
 //..
});
\end{lstlisting}
    Este método nos permite registrar comportamiento inmediatamente después de
    que la estructura del DOM este disponible en nuestro código
\end{frame}

\begin{frame}[fragile]{Métodos de Efectos} % (fold)
    Los métodos de evento nos permiten reaccionar a las entradas del usuario.
    En los métodos de efecto se hace con efecto, en lugar de ocultar y mostrar
    inmediatamente elementos lo hace con animación.
\begin{lstlisting}
this.toggleClass('arrow-down')
 .next()
 .slideToggle('fast');
\end{lstlisting}
    Este método realiza una rápida transición de deslizamiento en el elemento,
    permitiendo ocultar y mostrar alternativamente con cada invocación.
\end{frame}

\begin{frame}[fragile]{Métodos AJAX} % (fold)
    Muchos sitios web modernos emplean técnicas para carga el contenido cuando
    no se ah solicitado una actualización de la página. JQuery permite lograr
    esto con facilidad, los métodos de AJAX inician la solicitud de contenido y
    nos permite monitorear su progreso.
\begin{lstlisting}
$('#introduction > h2 a').click(function() {
 $('#introduction').load(this.href);
 return false;
\end{lstlisting}
    En la línea 2 el método \texttt{.load} nos permite obtener otro documento HTML
    desde el servidor e insertarlo en el documento actual,  todo ello con una
    línea de código.
\end{frame}

\end{document}
