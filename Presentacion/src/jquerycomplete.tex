\input{preambulo.tex}

% Titulo
\title[WebDesign] {Diseño Web | Tecnologías Involucradas}
\author[Walter Vargas]{ info@covetel.com.ve \inst{1}}
\subtitle{JQuery}
\institute[covetel.com.ve]{ \inst{1} Cooperativa Venezolana de Tecnologías Libres R.S. }
\date

\begin{document}

\begin{frame} % (fold)
    \titlepage 
\end{frame}

\begin{frame} % (fold)
    \tableofcontents
\end{frame}

\section{Anatomía de un Script de JQuery}

\subsection{Tablas Dinámicas de Contenido} % (fold)

\begin{frame}{JQuery} % (fold)
    JQuery es una biblioteca que contiene una amplia variedad de métodos y
    funciones de los cuales sera muy útil saber las categorías
    básicas y como utilizarlas para nuestro beneficio.\\[0.5cm]
\end{frame}

\begin{frame}[fragile]{Obtención de JQuery} % (fold)
    El sitio web oficial de JQuery siempre es el recurso más al día para el
    código y noticias relacionadas con la biblioteca. Para empezar necesitamos
    una copia de JQuery, que se puede descargar desde la página principal del
    sitio. Varias versiones de JQuery pueden estar disponibles en cualquier
    momento dado, la última versión sin comprimir será la más apropiada para
    nosotros.\\[0.5cm]
    Para usar JQuery no es necesaria su instalación sólo tenemos que colocarlo
    en nuestro sitio en un lugar público. Cada vez que necesitamos un metodo de
    JQuery simplemente lo referenciamos desde el archivo del documento
    HTML.\\[0.5cm]
\end{frame}

\begin{frame}[fragile]{Configuración del documento HTML} % (fold)
Existen tres componentes en el uso de JQuery:

\textbf{Componentes}
\begin{itemize}
    \item El Documento HTML.
    \item El Archivo CSS con los estilos.
    \item Los Archivos Javascript.
\end{itemize}

En este ejemplo se muestra como referenciar los componentes de JQuery.\\[0.5cm]

\end{frame}

\begin{frame}[fragile]{Configuración del documento HTML} % (fold)
\begin{lstlisting}
<html xmlns="http://www.w3.org/1999/xhtml" xml:lang:"en" lang:"en">
 <head>
  <meta http-equiv="Content-Type" content="text/html; charset=utf-8"/>
  <title>JQuery</title>
  <link rel="stylesheet" href="jquery.css" type="text/css" />
  <script src="jquery.js" type="text/javascript"></script>
  <script src="appjquery.js" type="text/javascript"></script>
  </head>
 <body>
 </body>
</html>
\end{lstlisting}

\end{frame}

\begin{frame}[fragile]{Escribiendo el Código JQuery} % (fold)
Nuestro código personalizado lo hemos incluido en el código HTML utilizando
\texttt{<script src="\/appjquery.js" type="text/javascript"\/></script>}. A pesar de lo mucho
que realiza tiene pocas lineas de código:
\end{frame}

\begin{frame}[fragile]{Escribiendo el Código JQuery} % (fold)
\begin{lstlisting}
JQuery.fn.toggleNext = function() {
 this.toggleClass('arrow-down')
 .next().slideToggle('fast');
}

$(document).ready(function () {
 $('div id="page-contents"></div>')
 .prepend('<h3>Page Contents</h3>')
 .append('<div></div>')
 .prependTo('body');
\end{lstlisting}
\end{frame}

\begin{frame}[fragile]{Escribiendo el Código JQuery} % (fold)
\begin{lstlisting}[firstnumber=last]
 $('#content h2').each(function(index) {
 var $chapterTitle = $(this);
 var chapterId = 'chapter~' + (index + 1);
 chapterTitle.attr('id', chapterId);
 $('<a></a>').text($chapterTitle.text())
  .attr({
   'title': 'Jump to' + $chapterTitle.text(), 
   'href': '#' + chapterId
  })
  .appendTo('#page-contents div');
  });
 \end{lstlisting}
\end{frame}

\begin{frame}[fragile]{Escribiendo el Código JQuery} % (fold)
 \begin{lstlisting}[firstnumber=last]
 $('#page-contents h3').click(function() {
  $(this).toggleNext();
 });

 $('#introduction > h2 a').click(function() {
  $('#introduction').load(this.href);
  return false;
 });
});
\end{lstlisting}
Este ejemplo crea una tabla dinámica para los usuarios.
\end{frame}

\subsection{Disección del Script} % (fold)

\begin{frame}[fragile]{Disección del Script} % (fold)
Este script ha sido elegido específicamente ya que ilustra la capacidad general
de la biblioteca JQuery e identifica las categorías de los métodos utilizados.
\end{frame}

\begin{frame}[fragile]{Selector de Expresiones} % (fold)
    Antes de que podamos actuar en un Documento HTML se debe localizar las
    partes pertinentes,  en el script algunas veces utilizamos un método simple
    para buscar un elemento:
\begin{lstlisting}
$('#introduction')
\end{lstlisting}
    Esta expresión se crea un nuevo objeto JQuery que hace referencia al
    elemento de ID \texttt{introduction}, algunas veces se requiere un selector
    más complejo:
\begin{lstlisting}
$('#introduction > h2 a')
\end{lstlisting}
   Aqui se obtiene un objeto JQuery que se refiere a los elementos
   descendientes de \texttt{<h2>} y que son hijos del elemento con ID
   \texttt{introduction}.   
\end{frame}

\begin{frame}[fragile]{Métodos de recorrido DOM} % (fold)
    A veces tenemos un objeto JQuery que ya hace referencia a un conjunto de
    elementos DOM,  pero tenemos que realizar una acción en uno diferente,  en
    estos casos los métodos de recorrido DOM son útiles, esto lo podemos ver en
    parte de nuestro script:
\begin{lstlisting}
this.toggleClass('arrow-down')
 .next()
 .slideToggle('fast');
\end{lstlisting}
    Debido al contexto de esta pieza de código, la palabra reservada
    \texttt{this} se refiere al objeto JQuery en este caso el objeto apunta la
    etiqueta \texttt{<h3>}. La llamada al método \texttt{.toggleClass} manipula
    este elemento de partida,  el método \texttt{.next()} cambia el
    funcionamiento del elemento y
    \texttt{.slideToggle} llama las acciones del elemento \texttt{<div>}
\end{frame}

\begin{frame}[fragile]{Métodos de Eventos} % (fold)
    Aun cuando podemos modificar la página a nuestra voluntad necesitamos
    métodos que reaccionen a eventos causados por los usuarios en el momento
    adecuado:
\begin{lstlisting}
$('#introduction > h2 a').click(function() {
 $('#introduction').load(this.href);
 return false;
});
\end{lstlisting}
    En este fragmento se registra un controlador que se ejecutará cada vez que
    se hace click en la etiqueta de anclaje seleccionado.
\end{frame}

\begin{frame}[fragile]{Métodos de Eventos} % (fold)
\begin{lstlisting}
$(document).ready(function() {
 //..
});
\end{lstlisting}
    Este método nos permite registrar comportamiento inmediatamente después de
    que la estructura del DOM este disponible en nuestro código
\end{frame}

\begin{frame}[fragile]{Métodos de Efectos} % (fold)
    Los métodos de evento nos permiten reaccionar a las entradas del usuario.
    En los métodos de efecto se hace con efecto, en lugar de ocultar y mostrar
    inmediatamente elementos lo hace con animación.
\begin{lstlisting}
this.toggleClass('arrow-down')
 .next()
 .slideToggle('fast');
\end{lstlisting}
    Este método realiza una rápida transición de deslizamiento en el elemento,
    permitiendo ocultar y mostrar alternativamente con cada invocación.
\end{frame}

\begin{frame}[fragile]{Métodos AJAX} % (fold)
    Muchos sitios web modernos emplean técnicas para carga el contenido cuando
    no se ah solicitado una actualización de la página. JQuery permite lograr
    esto con facilidad, los métodos de AJAX inician la solicitud de contenido y
    nos permite monitorear su progreso.
\begin{lstlisting}
$('#introduction > h2 a').click(function() {
 $('#introduction').load(this.href);
 return false;
\end{lstlisting}
    En la línea 2 el método \texttt{.load} nos permite obtener otro documento HTML
    desde el servidor e insertarlo en el documento actual,  todo ello con una
    línea de código.
\end{frame}

\section{Selector de Expresiones}

\subsection{Selectores CSS} % (fold)

\begin{frame}[fragile]{Elemento: T} % (fold)
Todos los elementos que tienen un nombre de etiqueta para T.
\textbf{Ejemplo}
\begin{itemize}
    \item \$('div') \textit{: selecciona todos los elementos con un nombre de etiqueta
    \texttt{div} en el documento.} 
    \item \$('em') \textit{: selecciona todos los elementos con un nombre de etiqueta
    \texttt{em} en el documento.}
\end{itemize}
JQuery utiliza JavaScript \texttt{getElementByTagName()} para los selectores de
nombre de etiqueta.
\end{frame}

\begin{frame}[fragile]{ID: \#myid} % (fold)
El único elemento con un ID igual a \texttt{myid}.

\textbf{Ejemplo}
\begin{itemize}
    \item \$('\#myid') \textit{: selecciona el unico elemento con \texttt{id = 'myid'},
    independientemente de su nombre de etiqueta. }
    \item \$('p\#myid') \textit{: selecciona un solo párrafo con \texttt{id} de
    \texttt{'myid'}, el único elemento \texttt{<p id = 'myid'>}. }
\end{itemize}
Cada nombre asignado a un \texttt{id} debe ser utilizado sólo una vez en un
documento.

Para un selector \texttt{id} JQuery utiliza el \texttt{getElementById()}.
\end{frame}

\begin{frame}[fragile]{Clase: .myclass} % (fold)
Todos los elementos que tienen una class \texttt{myclass}.
\textbf{Ejemplo}
\begin{itemize}
    \item \$('.myclass') \textit{: selecciona todos los elementos con una clase
    \texttt{myclass}.} 
    \item \$('p.myclass') \textit{: selecciona todos los párrafos con una clase 
    \texttt{myclass}.} 
    \item \$('.myclass.otherclass') \textit{: selecciona todos los elementos
    con una clase \texttt{myclass} y \texttt{otherclass} }
\end{itemize}
En términos de velocidad,  el ejemplo 2 es preferible al 1,  ya que el primero
utiliza el nativo \texttt{getElementsByTagName()} función de JavaScript para
filtrar la búsqueda, y las miradas de la clase dentro del subconjunto
emparejado de elementos DOM. 
\end{frame}

\begin{frame}[fragile]{Descendiente: E F} % (fold)
Todos los elementos encontrados por F que son descendientes de uno de los
elementos coincidentes por E.

\textbf{Ejemplo}
\begin{itemize}
    \item \$('\#container p') \textit{: selecciona todos los elementos
    encontrados por \texttt{<p>} que son descendientes de un elemento que tiene
    como \texttt{id} \texttt{container}.}
    \item \$('a img') \textit{: selecciona todos los elementos encontrados por
    \texttt{<img>} que son descendientes de un elemento coincidente con
    \texttt{<a>}.}
\end{itemize}
Un descendiente de un elemento puede ser un hijo, nieto,  bisnieto y así
sucesivamente. Por ejemplo en el siguiente código HTML,  el elemento
\texttt{<img>} es
descendiente de los elementos \texttt{<span>}, \texttt{<p>},  \texttt{<div
id="\/inner"\/>} y
\texttt{<div id="\/container"\/>}:

\begin{lstlisting}
<div id="container">
  <div id="inner">
    <p>
      <span><img src="example.jpg" alt="" /></span>
    </p>
  </div>
</div>
\end{lstlisting}
\end{frame}


\begin{frame}[fragile]{Hijo: E > F} % (fold)
Todos los elementos encontrados por F que son hijos de un elemento acompañado
por E.
\textbf{Ejemplo}
\begin{itemize}
    \item \$('li>ul') \textit{: selecciona todos los elementos que comienzan
    por \texttt{<ul>} que son hijo de un elemento acompañado por \texttt{<li>}}
    \item \$('p>code') \textit{: selecciona todos los elementos encontrados por
    \texttt{<code>} que son hijos de un elemento acompañado por \texttt{<p>}}
\end{itemize}
El combinador de hijo puede ser pensado de una forma mas especifica para
obtener los descendientes seleccionando sólo el descendiente de primer nivel,
en el siguiente código HTML el elemento \texttt{<img> es un hijo único del
elemento \texttt{<span>}}
\begin{lstlisting}
<div id="container">
  <div id="inner">
    <p>
      <span><img src="example.jpg" alt="" /></span>
    </p>
  </div>
</div>
\end{lstlisting}
\end{frame}

\begin{frame}[fragile]{Hermanos adyacentes: E + F} % (fold)
Todos los elementos encontrados por F seguidos inmediatamente por un Elemento E
y que tengan un mismo padre.
\textbf{Ejemplo}
\begin{itemize}
    \item \$('ul + p') \textit{: selecciona todos los elementos por
    \texttt{<p>} que están inmediatamente después de un elemento \texttt{<ul>}.}
    \item \$('strong + em') \textit{: selecciona todos los elementos
    encontrados por \texttt{<em>} que están después de un elemento
    \texttt{<strong>}.}
\end{itemize}
Un punto importante a considerar tanto con el combinar \texttt{+} como con el
\texttt{~} es que solo seleccionan los hermanos.
\begin{lstlisting}
<div id="container">
  <ul>
    <li></li>
    <li></li>
  </ul>
    <p>
      <img />
    </p>
</div>
\end{lstlisting}
\end{frame}

\begin{frame}[fragile]{Hermanos adyacentes: E + F} % (fold)
Del ejemplo anterior:
\begin{itemize}
    \item \$('ul + p') \textit{: selecciona \texttt{<p>} ya que esta
    inmediatamente después de \texttt{<ul>} y los dos elementos comparten el
    mismo padre \texttt{<div id="\/container"\/>}.}
    \item \$('ul + img') \textit{: no selecciona nada porque \texttt{<ul>} es
    un nivel mas alto que \texttt{<img>} en el árbol DOM.}
    \item \$('li + img') \textit{: no selecciona nada porque \texttt{<li>} y
    \texttt{<img>} estan en el mismo nivel en el árbol DOM.}
\end{itemize}
\end{frame}

\begin{frame}[fragile]{Hermanos en General: E ~ F} % (fold)
Todos los elementos encontrados por F que están después de un elemento E y
tienen el mismo padre.
\textbf{Ejemplo}
\begin{itemize}
    \item \$('p ~ ul') \textit{: selecciona todos los elementos encontrados por
    \texttt{<ul>} que están después de un elemento \texttt{<p>}.}
    \item \$('code ~ code') \textit{: selecciona todos los elementos encontrados
    por \texttt{<code>} después de \texttt{<code>}.}
\end{itemize}
La diferencia ente el combinador \texttt{+} y el \texttt{~} es el alcance.
mientras que el combinador \texttt{+} alcanza los elementos inmediatamente
siguientes el combinador \texttt{~} se extiende a todos los elementos de un
mismo nivel.
\begin{lstlisting}
  <ul>
    <li class="first"></li>
    <li class="second"></li>
    <li class ="third"></li>
  </ul>
  <ul>
    <li class="fourth"></li>
    <li class="fifth"></li>
    <li class="sixth"></li>
  </ul>
\end{lstlisting}
\end{frame}

\begin{frame}[fragile]{Hermanos adyacentes: E + F} % (fold)
Del ejemplo anterior:
\begin{itemize}
    \item \$('li.first ~ li') \textit{: selecciona \texttt{<li class="second"\/>}
    and \texttt{<li class="third"\/>}.}
    \item \$('li.first + li') \textit{: selecciona \texttt{<li class="second"\/>}.}
\end{itemize}
\end{frame}

\begin{frame}[fragile]{Multiple Elementos: E, F, G} % (fold)
Selecciona todas los elementos encontrados por el selector de expresiones E, F o G.
\begin{itemize}
    \item \$('code,  em,  strong') \textit{: selecciona todos los elementos
    encontrados por \texttt{<code>} or \texttt{<em>} or \texttt{<strong>}.} 
    \item \$('p strong, .myclass') \textit{: selecciona todos los elementos por
    \texttt{<strong>} que son descendientes de un elemento encontrado
    \texttt{<p>} y tienen una clase \texttt{.myclass}.}
\end{itemize}
El combinador coma \texttt{(,)} es eficiente para seleccionar elementos
dispares.
\end{frame}

\begin{frame}[fragile]{Enésimo Hijo (:nth-child(n))} % (fold)
Todos los elementos que son enésimos hijos de sus padres.
\textbf{Ejemplo}
\begin{itemize}
    \item \$('li:nth-child(2)') \textit{: selecciona todos los elementos
    encontrados por \texttt{<li>} que son segundo hijo del padre.}
    \item \$('p:nth-child(5)') \textit{: selecciona todos los elementos
    encontrados por \texttt{<p>} que son quinto hijo de su padre.}
\end{itemize}
\begin{lstlisting}
  <div>
    <h2></h2>
    <p></p>
    <h2></h2>
    <p></p>
    <p></p>
  </div>
\end{lstlisting}
\begin{itemize}
    \item \$('p:nth(1)') \textit{: selecciona el segundo \texttt{<p>} porque la
    numeración para \texttt{:nth(n)} comienza en 0.} 
    \item \$('p:nth-child(1)') \textit{: no selecciona nada porque el primer
    elemento hijo del padre no es \texttt{<p>}.}
    \item \$('p:nth(2)') \textit{: selecciona el primer \texttt{<p>} porque es
    el segundo hijo del padre.}
\end{itemize}
\end{frame}

\begin{frame}[fragile]{Hijos} % (fold)
\begin{itemize}
    \item Primer Hijo (:first-child) \textit{: Todos los elementos que son
    primer hijo de su padres.}

\textbf{Ejemplo}
\begin{itemize}
    \item \$('li:first-child') \textit{: selecciona todos los elementos
    encontrados por \texttt{<li>} que son el primer hijo del padre.} 
\end{itemize}
    \item Ultimo Hijo (:last-child) \textit{: Todos los elementos que son el último hijo del padre.}

\textbf{Ejemplo}
\begin{itemize}
    \item \$('li:first-last') \textit{: selecciona todos los elementos
    encontrados por \texttt{<li>} que son el último hijo del padre.} 
\end{itemize}
    \item Único Hijo (:only-child) \textit{: Todos los elementos que son único
    hijo del padre.}

\textbf{Ejemplo}
\begin{itemize}
    \item \$('li:first-last') \textit{: selecciona todos los elementos
    encontrados por \texttt{<li>} que son único hijo del padre.} 
\end{itemize}
\end{itemize}
\end{frame}

\begin{frame}[fragile]{Diversos Selectores} % (fold)
\begin{itemize}
    \item No (:not(s)) \textit{: Todos los elementos que no coinciden con el
    selector \texttt{s}.}
    
    \textbf{Ejemplo}
    \begin{itemize}
        \item \$('li:not(.myclass)') \textit{: Selecciona todos los elemntos
        encontrados por \texttt{<li>} que no tienen \texttt{class="myclass"}.}
    \end{itemize}
    \item Vacio (:empty) \textit{: Todos los elementos que no tienen hijos
    (incluyendo los nodos de texto).}
    
    \textbf{Ejemplo}
    \begin{itemize}
        \item \$('p:empty') \textit{: Selecciona todos los elementos
        encontrados por \texttt{<p>} que no tienen hijos.}
    \end{itemize}
    \item Universal (:*)\textit{: Todos los elementos.}
    
    \textbf{Ejemplo}
    \begin{itemize}
        \item \$('*') \textit{: Selecciona todos los elementos en el documento.}
    \end{itemize}
\end{itemize}
\end{frame}

\subsection{Selectores XPath} % (fold)

\begin{frame}[fragile]{Descendientes: E//F} % (fold)
Todos los elementos encontrados por F que son descendientes de un elemento
encontrado por E.
\textbf{Ejemplo}
\begin{itemize}
    \item \$('div//code'): \textit{: selecciona todos los elementos encontrados
    do \texttt{<code>} que son descendientes de un elemento encontrado por
    \texttt{<div>}}
\end{itemize}
Este selector de descendientes de XPath funciona igual que el selector de
descendiente de CSS, salvo que en la version XPath se puede especificar que
inicie desde la raíz del documento.
\end{frame}

\begin{frame}[fragile]{Hijos: E/F} % (fold)
Todos los elementos encontrados por F que son hijos de un elemento
conincidente con E.
\textbf{Ejemplo}
\begin{itemize}
    \item \$('/docroot/el') \textit{: Selecciona todos los elementos
    encontrados por \texttt{<el>} que son hijos de un elemento coincidente con
    \texttt{<docroot>}.}
\end{itemize}
El selector de XPath hijo es una alternativa para el selector CSS hijo. Si el
selector de expresión comienza con un slash como es el caso del ejemplo,  el
selector inmediatamente después del slash debe estar en la raíz del documento. 
\end{frame}

\begin{frame}[fragile]{Padres: E/..} % (fold)
Todos los elementos que son padres de un elemento que coincide con E.
\textbf{Ejemplo}
\begin{itemize}
    \item \$('.myclass/..') \textit{: Selecciona los elementos padre de todos
    los elementos que tengan una clase \texttt{myclass}.}
    \item \$('.myclass/../p') \textit{: Selecciona los elementos que coinciden
    con \texttt{<p>} que son hijos de un elemento que tiene una clase
    \texttt{myclass}.}
\end{itemize}
\begin{lstlisting}
  <div>
    <p id="firstp"></p>
    <div id="subdiv"></div>
    <p id="secondp">
      <span class="myclass"></span>
    </p>
  </div>
  <div>
    <p></p>
  </div>
\end{lstlisting}
\end{frame}

\begin{frame}[fragile]{Padres: E/..} % (fold)
En el código anterior:
\begin{itemize}
    \item \$('span.myclass/..') \textit{: selecciona \texttt{<p id="secondp"\/>} porque este es el padre de \texttt{<span class="myclass"\/>}.}
    \item \$('\#firstp/../') \textit{: selecciona \texttt{<p id="firstp"\/>}, \texttt{<div id="subdiv"\/>} y \texttt{<p id="secondp"\/>} porque el selector:}
    \begin{itemize}
        \item a \textit{: Comienza con \texttt{<p id="firstp"\/>}.}
        \item b \textit{: Atraviesa un nivel en el árbol DOM.}
        \item c \textit{: Selecciona todos los hijos de \texttt{<div>}.}
    \end{itemize}
\end{itemize}
\end{frame}

\begin{frame}[fragile]{Contiene: [F]} % (fold)
Todos los elementos que coinciden con F.
\textbf{Ejemplo}
\begin{itemize}
    \item \$('p[.myclass]') \textit{: selecciona todos los elementos
    encontrados por \texttt{<p>} que contienen un elemento con una clase
    \texttt{myclass}.}
\end{itemize}
\end{frame}

\subsection{Selectores de Atributos} % (fold)

\begin{frame}[fragile]{Valores de Atributos} % (fold)
\begin{itemize}
\item Tiene Atributos: [@foo] \textit{Todos los elementos que tienen el
atributo foo.}
    
    \textbf{Ejemplo}
    \begin{itemize}
        \item \$('p[@class]') \textit{: selecciona todos los elementos encontrados
    por \texttt{<p>} que tiene una class como atributo.}
    \end{itemize}
\item Valores de Atributos no iguales: [@foo!=bar] \textit{Todos los elementos en los que el atributo foo no tiene exactamente el valor
bar.}
    
    \textbf{Ejemplo}
    \begin{itemize}
        \item \$('input[@name!=myname]') \textit{: selecciona todos los elementos
    coincidentes con \texttt{<input>} en los que el atributo \texttt{name} no
    tiene como valor \texttt{myname}.}
    \end{itemize}

\item Comienzo del valor de los Atributos: [@foo\^=bar] \textit{Todos los
elementos que tengan el atributo foo y su valor comience con el string bar.}
    
    \textbf{Ejemplo}
    \begin{itemize}
        \item \$('input[@name\^=my]') \textit{: selecciona todos los elementos
    coincidentes con \texttt{<input>} cuyo valor del atributo \texttt{name}
    comience con \texttt{my}.}
    \end{itemize}
\end{itemize}
\end{frame}

\begin{frame}[fragile]{Valores de Atributos} % (fold)
\begin{itemize}
\item Final del valor de los Atributos: [@foo\$=bar] \textit{Todos los elementos
que tengan el atributo foo y su valor termina con el string bar.}
    
    \textbf{Ejemplo}
    \begin{itemize}
        \item \$('a[@href\$=index.html]') \textit{: selecciona todos los elementos
    coincidentes con \texttt{<a>} en los que el valor del atributo
    \texttt{href} termine con \texttt{index.html}.}
    \end{itemize}

\item Atributos en los que el valor contiene: [@foo*=bar] \textit{Todos los
elementos en los que el atributo foo contiene en su valor el string bar.}
    
    \textbf{Ejemplo}
    \begin{itemize}
        \item \$('a[href*=example.com]') \textit{: selecciona todos los elementos
    coincidentes con \texttt{<a>} en los que el valor del atributo
    \texttt{href} contiene el string \texttt{example.com}.}
    \end{itemize}
\end{itemize}
\end{frame}

\subsection{Selectores de Formulario} % (fold)

\begin{frame}[fragile]{Selectores de Formulario} % (fold)
Los siguientes selectores se pueden utilizar para acceder a los elementos del
formulario en cualquier estado.
\begin{itemize}
        \item Todos los elementos del formulario \texttt{(<input>, <select>,
        <textarea>, <button>)}.
        \item Todos los campos de texto \texttt{(<input type="text"\/>)}.
        \item Todos los campos de password \texttt{(<input type="password"\/>)}.
        \item Todos los input submit y elementos button \texttt{(<input
        type="submit"\/>, <button>)}.
        \item Todos los input de imagen \texttt{(<input type="image"\/>)}.
        \item Todos los elementos de interfaz de usuario que estén habilitadas.
        \item Todos los elementos de interfaz de usuario checkboxes and radio
        buttons.
\end{itemize}
\end{frame}

\subsection{Selectores Personalizados} % (fold)

\begin{frame}[fragile]{Elemento Par,  Elemento Impar, Enésimo Elemento}
\begin{itemize}
\item Elemento Par (:even) Elemento Impar (:odd) 

\textit{Todos los elementos con indice par: 
  :even
Todos los elementos con un indice impar:
  :odd.}

\textbf{Ejemplo}  
\begin{itemize}
    \item \$('li:even') \textit{: selecciona todos los elementos que coinciden
    con \texttt{li} que tienen un valor indice par.}
    \item \$('li:even') \textit{: selecciona todos los elementos que coinciden
    con \texttt{<tr>} que tienen un valor indice impar}
\end{itemize}

\item Enésimo Elemento (:eq(n), :nth(n)) 

\textit{El elemento con un valor de indice igual a n.}

\textbf{Ejemplo}
\begin{itemize}
    \item \$('li:eq(2)') \textit{: selecciona el tercer elemento \texttt{<li>} .}
    \item \$('p:nth(1)') \textit{: selecciona el segundo elemento \texttt{<p>} .}
\end{itemize}
\end{itemize}
\end{frame}

\begin{frame}[fragile]{Mayor que,  Menor que, Primero, Ultimo} % (fold)
\begin{itemize}
    \item Mayor que :gt(n) \textit{Todos los elementos con un indice mayor que N}

        \textbf{Ejemplo}
        \begin{itemize}
            \item \$('li:gt(1)') \textit{: selecciona todos los elementos
        encontrados por \texttt{<li>} a parir del segundo.}
        \end{itemize}
    \item Menor que :lt(n) \textit{Todos los elementos con un indice menor que N}

        \textbf{Ejemplo}
        \begin{itemize}
            \item \$('li:lt(3)') \textit{: selecciona todos los elementos
            encontrados por \texttt{<li>} anteriores del tercero,  en otras palabras los dos primeros.}
        \end{itemize}
    \item Primero :first \textit{La primera instancia de un elemento}

        \textbf{Ejemplo}
        \begin{itemize}
            \item \$('a:first') \textit{: selecciona el primer elemento \texttt{<a>}.}
        \end{itemize}
    \item Ultimo :last \textit{La ultima instancia de un elemento}

        \textbf{Ejemplo}
        \begin{itemize}
            \item \$('\#container .myclass:last') \textit{: selecciona el último elemento con una clase \texttt{myclass} y es descendiente de un
        elemento con \texttt{id="container"}.}
        \end{itemize}
\end{itemize}
\end{frame}


\begin{frame}[fragile]{Padre,  Que Contiene, Visible, Oculto} % (fold)
\begin{itemize}
    \item Padre :parent.
    
    \textit{Todos los elementos que son padres de otro elemento.}
    
    \textbf{Ejemplo}
        \begin{itemize}
            \item \$('td:parent') \textit{: selecciona todos los elementos encontrados por \texttt{<td>} que son padres de otro elemento.}
        \end{itemize}

    \item Que Contiene :contains(text).
    
    \textit{Todos los elementos que contienen el texto especificado.}
    
    \textbf{Ejemplo}
        \begin{itemize}
            \item \$('li:contains(second)') \textit{: selecciona todos los
            elementos encontrados por \texttt{<li>} que son contienen el texto
            second.}
        \end{itemize}
\end{itemize}
\end{frame}

\begin{frame}[fragile]{Padre,  Que Contiene, Visible, Oculto} % (fold)
\begin{itemize}
    \item Visible :visible.
    
    \textit{Todos los elementos que son Visibles.}

    \textbf{Ejemplo}
        \begin{itemize}
            \item \$('td:visible') \textit{: selecciona todos los elementos
            encontrados por \texttt{<td>} que son visibles.}
        \end{itemize}

    \item Oculto :hidden.
    
    \textit{Todos los elementos que son ocultos.}
    
    \textbf{Ejemplo}
        \begin{itemize}
            \item \$('input:hidden') \textit{: selecciona todos los elementos
            encontrados por \texttt{<input>} que están ocultos.}
        \end{itemize}
\end{itemize}
\end{frame}

\section{Recorrido de los métodos DOM}

\subsection{\$()} % (fold)

\begin{frame}[fragile]{\$()} % (fold)
Crea un nuevo objeto JQuery encontrando elementos en el DOM.
\begin{lstlisting}
  $(selector[,  context])
  $(element)
  $(elementArray)
  $(Object)
  $(html)
\end{lstlisting}
\textbf{Ejemplo}
\begin{itemize}
    \item selector \textit{: una cadena que contiene una expresión de selección.}
    \item context \textit{: parte del árbol DOM dentro de la cual se busca.}
    \item element \textit{: un elemento DOM para envolver en un objeto JQuery.}
    \item elementArray \textit{: un array contiene un conjunto de elementos DOM
    para envolver en un objeto JQuery.}
    \item object \textit{: un objeto JQuery existente para clonar.}
    \item html \textit{: una cadena que contiene un código HTMK que describen
    el nuevo elemento DOM a crear.}
\end{itemize}
\end{frame}

\subsection{Métodos de Filtrado} % (fold)

\begin{frame}[fragile]{.filter()} % (fold)
Reduce el conjunto de elementos encontrados con aquellos que coincidan con el
selector.
\textbf{Ejemplo}
\begin{itemize}
    \item \$('.filter(selector)') \textit{ selector= una cadena que contiene
    una expresión de selección que coincida con los elementos.}
    \item \$('.filter(function)') \textit{ function= una función usada como
    prueba para cada elemento del conjunto.}
\end{itemize}
\begin{lstlisting}
  <ul>
    <li><strong> list </strong> item 1 - one strong </li>
    <li><strong> list </strong> item <strong>2</strong> -
    two<span>strongs</span></li>
    <li>list item 3</li>
    <li>list item 4</li>
    <li>list item 5</li>
    <li>list item 6</li>
  </ul>    
\end{lstlisting}
\end{frame}

\begin{frame}[fragile]{.filter()} % (fold)
\begin{lstlisting}
$('li').filter(function(index) {
  return \$("strong",  this).length ==1;
})
\end{lstlisting}
El resultado de esta expresión será el primer elemento de la lista, ya que
contiene exactamente una etiqueta \texttt{<strong>} dentro de la función de
filtro. \texttt{this} se refiere a cada elemento DOM a su vez. El parámetro
pasado a la función nos dice que es el elemento DOM a ser comparado.
\end{frame}

\begin{frame}[fragile]{.not()} % (fold)
\begin{itemize}
    \item .not(selector) \textit{ selector= una cadena que contiene una
    expresión para que coincida con los elementos.}
    \item .not(elements) \textit{ elements= uno o mas elementos DOM para
    eliminar del conjunto.}
\end{itemize}
\begin{lstlisting}
<ul>
  <li>list item 1</li>
  <li>list item 2</li>
  <li id="notli">list item 3</li>
  <li>list item 4</li>
  <li>list item 5</li>
</ul>
\end{lstlisting}
\begin{lstlisting}
$('li').not(document.getElementById('notli'))
\end{lstlisting}
Esta expresión obtiene un objeto JQuery con los items 1, 2, 4 y 5.
\end{frame}

\subsection{Recorrido del árbol de métodos} % (fold)

\begin{frame}[fragile]{.find()} % (fold)
\begin{itemize}
    \item .find(selector) \textit{ selector= un string que contiene un selector
    que coincida con los elementos.}
\end{itemize}
\begin{lstlisting}
<ul class="level-1">
  <li class="item-i">I</li>
  <li class="item-ii">II
    <ul class="level-2">
      <li class="item-a">A</li>
      <li class="item-b">B
        <ul class="level-3">
          <li class="item-1">1</li>
          <li class="item-2">2</li>
          <li class="item-3">3</li>
        </ul>
      </li>
      <li class="item-c">C</li>
    </ul>
  </li>
  <li class="item-iii">III</li>
</ul>
\end{lstlisting}
\begin{lstlisting}
$('li.item-ii').find('li')
\end{lstlisting}
El resultado de esta llamada es un objeto JQuery que contiene los items A, B, 1, 2, 3 and C.\\[0.9cm] 
\end{frame}

\begin{frame}[fragile]{.children()} % (fold)
\begin{itemize}
    \item .children([selector]) \textit{ selector=un string que contiene un selector que coincida con los elementos }
\end{itemize}
\begin{lstlisting}
<ul class="level-1">
  <li class="item-i">I</li>
  <li class="item-ii">II
    <ul class="level-2">
      <li class="item-a">A</li>
      <li class="item-b">B
        <ul class="level-3">
          <li class="item-1">1</li>
          <li class="item-2">2</li>
          <li class="item-3">3</li>
        </ul>
      </li>
      <li class="item-c">C</li>
    </ul>
  </li>
  <li class="item-iii">III</li>
</ul>
\end{lstlisting}
\begin{lstlisting}
$('ul.level-2').children()
\end{lstlisting}
El resultado de esta llamada es un objeto JQuery que contiene los items A, B y C. \\[0.9cm]
\end{frame}

\begin{frame}[fragile]{.parents()} % (fold)
\begin{itemize}
    \item .parents([selector]) \textit{ selector=un string que contiene un selector que coincida con los elementos }
\end{itemize}
\begin{lstlisting}
<ul class="level-1">
  <li class="item-i">I</li>
  <li class="item-ii">II
    <ul class="level-2">
      <li class="item-a">A</li>
      <li class="item-b">B
        <ul class="level-3">
          <li class="item-1">1</li>
          <li class="item-2">2</li>
          <li class="item-3">3</li>
        </ul>
      </li>
      <li class="item-c">C</li>
    </ul>
  </li>
  <li class="item-iii">III</li>
</ul>
\end{lstlisting}
\begin{lstlisting}
$('li.item-a').parents()
\end{lstlisting}
El resultado de esta llamada es un objeto JQuery que contiene la lista level-2, item ii y la lista level-1.\\[0.9cm]
\end{frame}

\begin{frame}[fragile]{.prev()} % (fold)
\begin{itemize}
    \item .prev([selector]) \textit{ selector=un string que contiene un selector que coincida con los elementos }
\end{itemize}
\begin{lstlisting}
<ul>
  <li>list item 1</li>
  <li>list item 2</li>
  <li class="third-item">list item 3</li>
  <li>list item 4</li>
  <li>list item 5</li>
</ul>
\end{lstlisting}
\begin{lstlisting}
$('li.third-item').prev()
\end{lstlisting}
El resultado de esta llamada es un objeto JQuery que contiene item 2.
\end{frame}

\subsection{Varios métodos de recorrido} % (fold)

\begin{frame}[fragile]{.add()} % (fold)
\begin{itemize}
    \item .add(selector) \textit{ selector=un string que contiene un selector que coincida con los elementos adicionales}
    \item .add(elements) \textit{ elements=uno o mas elementos para agregar al
    conjunto de elementos encontrados.}
    \item .add(html) \textit{ html=un fragmento HTML para agregar al conjunto
    de elementos encontrados.}
\end{itemize}
\begin{lstlisting}
<ul>
  <li>list item 1</li>
  <li>list item 2</li>
  <li>list item 3</li>
</ul>
<p>un parrafo</p>
\end{lstlisting}
\begin{lstlisting}
$('li').add('p')
\end{lstlisting}
El resultado de esta llamada es un objeto JQuery que contiene los 4 elementos.
\end{frame}

\begin{frame}[fragile]{.is()} % (fold)
\begin{itemize}
    \item .is(selector) \textit{ selector=un string que contiene un selector que coincida con los elementos adicionales}
\end{itemize}
\begin{lstlisting}
<ul>
  <li>list <strong> item 1 </strong></li>
  <li><span>list item 2</span></li>
  <li>list item 3</li>
</ul>
\end{lstlisting}
\begin{lstlisting}
$('ul').click(function(event) {
  if ($(event.target).is('li') ) {
    $(even.target).remove();
  }
});
\end{lstlisting}
Ahora cuando el usuario presione click en la palabra list en el primer item o
donde sea en el tercer item, el elemento que hace click en la lista se
eliminara del documento.
\end{frame}

\section{Métodos de manipulación DOM} % (fold)

\subsection{Atributos Generales} % (fold)

\begin{frame}[fragile]{.attr(attribute)} % (fold)
\begin{itemize}
    \item .attr(attribute) \textit{ attribute= el nombre del atributo a
    obtener.}
    \item .attr(attribute,  value) \textit{ value= un valor establecido para el atributo.}
    \item .attr(map) \textit{ map= un mapa de pares atributo-valor para
    establecer.}
    \item .attr(attribute,  function) \textit{ function= una función que
    devuelve el valor a establecer.}
\end{itemize}
\begin{lstlisting}
<img id="greatphoto" src="brush-seller.jpg" alt="brush seller" />
\end{lstlisting}
\begin{lstlisting}
$('#greatphoto').attr('alt', 'Beijing Brush Seller');
\end{lstlisting}
Cambia el atributo alt señalando el atributo alt y estableciendo el nuevo valor después de la coma dentro del paréntesis.
\end{frame}

\begin{frame}[fragile]{.attr()} % (fold)
Mediante el uso de una función para establecer los atributos, podemos concatenar un nuevo valor con un valor existente.
\begin{lstlisting}
$('#greatphoto').attr({alt: function() {return 'Beijing' + this.alt}, title:
function() {return 'Beijing' + this.alt + ' - photo by Kelly Clarck'}});
\end{lstlisting}
El uso de una función puede ser aún mas útil cuando se aplican los atributos de
elementos multiples.
\end{frame}

\begin{frame}[fragile]{.removeAttr()} % (fold)
\begin{itemize}
    \item .removeAttr(attribute) \textit{ attribute= un atributo.}
\end{itemize}
El método .removeAttr utiliza la función JavaScript removeAttribute, pero tiene
la ventaja de poder ser encadenado a una expresión de selección JQuery.
\end{frame}

\subsection{Propiedades de Estilo} % (fold)

\begin{frame}[fragile]{.css()} % (fold)
\begin{itemize}
    \item .css(property,  value) \textit{ property= un nombre de propiedad CSS.    
    value= un valor establecido para la propiedad.}
    \item .css(map) \textit{ map= un mapa de pares propiedad-valor para
    establecer.}
    \item .css(property, function) \textit{ function= una función que devuelve el valor a establecer.}
\end{itemize}
\begin{lstlisting}
$('div.example').css('width', function(index) {
  return index * 50;
});
\end{lstlisting}
Este ejemplo establece la anchura de los elementos que coinciden con lo valores
progresivamente más grandes.
\end{frame}

\begin{frame}[fragile]{Dimensiones} % (fold)
\begin{itemize}
    \item .height(value) \textit{ value= un entero representando el número de pixeles o un entero junto con una unidad opcional de medida.}
    \item .width(value) \textit{ value= un entero representado el número de o un entero junto con una unidad opcional de medida.}
\end{itemize}
\end{frame}

\subsection{Atributos de Clase} % (fold)

\begin{frame}[fragile]{Atributos de Clase} % (fold)
\begin{itemize}
    \item .addClass(class) \textit{ class= uno o mas nombres de clase que se agrega al atributo de clase o a cada elemento emparejado.}
    \item .removeClass(class) \textit{ class= uno o mas nombres de clase que se eliminan del atributo de clase o de cada elemento emparejado.}
\begin{lstlisting}
$('p').removeClass('myclass noclass').addClass('yourclass')
\end{lstlisting}
Aquí las clases \texttt{myclass} y \texttt{noclass} se eliminan de los
párrafos, mientras \texttt{yourclass} es añadida.
    \item .toggleClass(class) \textit{ function= un nombre de clase para ser activado en el atributo de clase de cada elemento en el conjunto combinado.}
\begin{lstlisting}
<div class="tumble">Some text</div>
\end{lstlisting}
\begin{lstlisting}
$('div.tumble').toggleClass('bounce')
\end{lstlisting}
\begin{lstlisting}
<div class="tumble bounce">Some text</div>
\end{lstlisting}
\end{itemize}
\end{frame}

\subsection{Reemplazo DOM} % (fold)

\begin{frame}[fragile]{Reemplazo DOM} % (fold)
\begin{itemize}
    \item .html(HTML) \textit{ HTML= una cadena HTML para establecer el contenido de cada elemento encontrado.} 
    \begin{lstlisting}
    <div class="demo-container">
      <div class"demo-box">Demostration Box</div>
    </div>
    \end{lstlisting}
    \begin{lstlisting}
    $('div.demo-container'>.html('<p>All new content. <em>You bet!</em></p>'));
    \end{lstlisting}
    \begin{lstlisting}
    <div class="demo-container">
      <p>All new content. <em>You Bet!</em></p>
      <div class"demo-box">Demostration Box</div>
    </div>
    \end{lstlisting}
\end{itemize}
\end{frame}

\begin{frame}[fragile]{Métodos} % (fold)
\begin{itemize}
    \item .text(text) \textit{ text= una cadena de texto para establecer el contenido de cada elemento encontrado.} 
    \begin{lstlisting}
    <div class="demo-container">
      <div class"demo-box">Demostration Box</div>
      <ul>
        <li>list item1</li>
        <li>list <strong>item</strong>2</li>
      </ul>
    </div>
    \end{lstlisting}
    \begin{lstlisting}
    $('div.demo-container').text('<p>This is a Test.</p>')
    \end{lstlisting}
    \begin{lstlisting}
    <div class="demo-container">
     <p>This is a test</p>
    </div>
    \end{lstlisting}
    \item .val(value) \textit{ value= una cadena de texto para establecer el valor de la propiedad de cada elemento encontrado.}
\end{itemize}
\end{frame}

\subsection{Inserciones DOM, Dentro} % (fold)

\begin{frame}[fragile]{Inserciones DOM, Dentro} % (fold)
\begin{itemize}
    \item .prepend(content) \textit{ content= un elemento, cadena HTML u objeto
    jQuery para insertar al principio del elemento establecido encontrado.} 
    \begin{lstlisting}
    <div class="demo-container">
      <div class"demo-box">Demostration Box</div>
    </div>
    \end{lstlisting}
    \begin{lstlisting}
    $('div.demo-box').prepend('<div class="insertion">This text
    was<strong>inserted</strong></div>');
    \end{lstlisting}
    \item .prependTo(target) \textit{ target= un selector, elemento, cadena
    HTML, cadena u objeto JQuery; el conjunto combinado de elementos se
    insertan al principio del elemento especificado por este parámetro.} 
    \begin{lstlisting}
    <div class="demo-container">
      <div class"demo-box">Demostration Box</div>
    </div>
    \end{lstlisting}
    \begin{lstlisting}
    $('<div class="insertion">This text was<strong>inserted</strong></div>').prependTo('div.demo-box');
    \end{lstlisting}
\end{itemize}
\end{frame}

\begin{frame}[fragile]{Inserciones DOM, Dentro} % (fold)
\begin{itemize}
    \item .append(content) \textit{ content= un elemento, cadena HTML u objeto
    jQuery para insertar al final del elemento establecido encontrado.} 
    \begin{lstlisting}
    <div class="demo-container">
      <div class"demo-box">Demostration Box</div>
    </div>
    \end{lstlisting}
    \begin{lstlisting}
    $('div.demo-box').append('<div class="insertion">This text
    was<strong>inserted</strong></div>');
    \end{lstlisting}
    \item .appendTo(target) \textit{ target= un selector, elemento, cadena
    HTML, cadena u objeto JQuery; el conjunto combinado de elementos se
    insertan al final del elemento especificado por este parámetro.} 
    \begin{lstlisting}
    <div class="demo-container">
      <div class"demo-box">Demostration Box</div>
    </div>
    \end{lstlisting}
    \begin{lstlisting}
    $('<div class="insertion">This text was<strong>inserted</strong></div>').appendTo('div.demo-box');
    \end{lstlisting}
\end{itemize}
\end{frame}

\subsection{Inserciones DOM, Fuera} % (fold)

\begin{frame}[fragile]{Inserciones DOM, Fuera} % (fold)
\begin{itemize}
    \item .insertBefore(content) \textit{ content= un selector o elemento que
    se insertara antes del conjunto encontrado.} 
    \begin{lstlisting}
    <div class="demo-container">
      <div class"demo-box">Demostration Box</div>
    </div>
    \end{lstlisting}
    \begin{lstlisting}
    $('<div class="insertion">This text was<strong>inserted</strong></div>').insertBefore('div.demo-box');
    \end{lstlisting}
    \item .insertAfter(content) \textit{ content= un selector o elemento que se insertara después del conjunto encontrado.} 
    \begin{lstlisting}
    <div class="demo-container">
      <div class"demo-box">Demostration Box</div>
    </div>
    \end{lstlisting}
    \begin{lstlisting}
    $('<div class="insertion">This text was<strong>inserted</strong></div>').insertAfter('div.demo-box');
    \end{lstlisting}
\end{itemize}
\end{frame}

\subsection{DOM Copia, DOM Eliminación} % (fold)

\begin{frame}[fragile]{DOM Copia, DOM Eliminación} % (fold)
\begin{itemize}
    \item .clone([deep]) \textit{ deep= Un booleano. Predeterminado verdadero. Si se establece falso el método copia sólo los elementos coincidentes con si mismo con exclusión de cualquier hijo o descendiente.} 
    \begin{lstlisting}
    <div class="demo-container">
      <div class"demo-box">Demostration Box</div>
    </div>
    \end{lstlisting}
    \begin{lstlisting}
    $('div.demo-box:last').clone().insertAfter('div.demo-box:last');
    \end{lstlisting}
    \item .remove([selector]) \textit{ selector= Un selector que filtra los elementos encontrado a ser removidos.} 
    \begin{lstlisting}
    <div class="demo-container">
      <div class"demo-box">Demostration Box</div>
    </div>
    \end{lstlisting}
    \begin{lstlisting}
    $('div.demo-box').remove('#temporary-demo-box')
    \end{lstlisting}
\end{itemize}
\end{frame}

\section{Métodos de Evento} % (fold)

\subsection{Controlador adjunto de eventos} % (fold)

\begin{frame}[fragile]{Controlador adjunto de eventos} % (fold)
\begin{itemize}
    \item .bind(evenType[, eventData, handler]) \textit{ evenType= Una cadena que contiene un tipo de evento JavaScript, sea click o submit.}

    \textit{ eventData= Un mapa de datos que pueden ser pasados por un   manejador de eventos.}

    \textit{ handler= Una función que se ejecutará cada vez que se dispare el
    evento.}
    \begin{lstlisting}
    $('#foo').bind('click', function() {
      alert('User clicked on foo');
    });
    \end{lstlisting}
    \item .trigger(evenType [, extraParameters]) \textit{ extraParameters= Un array de parámetros adicionales pasados a través de un manejador de eventos.} 
    \begin{lstlisting}
    $('#foo').bind('click' function() {
      alert($(this).text());
    });
    $('#foo').trigger('click');
    \end{lstlisting}
\end{itemize}
\end{frame}

\subsection{Cargar de Documentos} % (fold)

\begin{frame}[fragile]{Carga de Documentos} % (fold)
\begin{itemize}
    \item .load(handler) \textit{ handler= Una función que se ejecutará cada vez que se dispare el evento.}
    \begin{lstlisting}
      <img class="target" src="hat.gif" width="80" height="54" />
    \end{lstlisting}
    \begin{lstlisting}
    $('.target').load(function() {
      $(this).log('Load event was triggered.');
    });
    \end{lstlisting}
    \item .error(handler) \textit{ handler= Una función que se ejecutará cada vez que se dispare el evento.} 
    \begin{lstlisting}
      <img class="target" src="missing.gif" width="80" height="54" />
    \end{lstlisting}
    \begin{lstlisting}
    $('.target').error(function() {
      $(this).log('Error event was triggered.');
    });
    \end{lstlisting}
\end{itemize}
\end{frame}

\subsection{Eventos del Mouse} % (fold)

\begin{frame}[fragile]{Eventos del Mouse} % (fold)
\begin{itemize}
    \item .click(handler) \textit{ handler= Una función que se ejecutará cada vez que se dispare el evento.}
    \begin{lstlisting}
    $('.target').click(function() {
      $(this).log('Click event was triggered.');
    });
    \end{lstlisting}
    \item .mouseover(handler) \textit{ handler= Una función que se ejecutará cada vez que se dispare el evento.} 
    \begin{lstlisting}
    $('.target').mouseover(function() {
      $(this).log('Mouseover event was triggered.');
    });
    \end{lstlisting}
    \item .hover(handlerIn, handlerOut) \textit{ handlerIn= Una función que se
    ejecutará cuando el puntero del mouse entra en el elemento.} 

    \textit{ handlerOut= Una función que se ejecutará cuando el puntero del mouse sale del elemento.} 
    \begin{lstlisting}
    $('.target').hover(function() {
      $(this).log('Hover event was triggered entering');
    }function() {
      $(this).log('Hover event was triggered leaving');
    });
    \end{lstlisting}
\end{itemize}
\end{frame}

\begin{frame}[fragile]{Eventos del Mouse} % (fold)
\begin{itemize}
    \item .mousemove(). 
    \item .dblclick(). 
    \item .mousemove(). 
    \item .toggle(). 
    \item .mouseup(.) 
    \item .mousedown().
\end{itemize}
\end{frame}

\subsection{Eventos del Formulario} % (fold)

\begin{frame}[fragile]{Eventos del Formulario} % (fold)
\begin{itemize}
    \item .focus(handler) \textit{ handler= Una función que se ejecutará cada vez que se dispare el evento.}
    \begin{lstlisting}
    $('.target').focus(function() {
      $(this).log('Focus event was triggered.');
    });
    \end{lstlisting}
    \item .blur().
    \item .change().
    \item .select().
    \item .submit(handler) \textit{ handler= Una función que se ejecutará cada vez que se dispare el evento.} 
    \begin{lstlisting}
    <form class="target" action="foo.html">
      <input type="text" /><input type="submit" value="Go" />
    </form>
    <div class="trigger button">Button</div>
    \end{lstlisting}
    \begin{lstlisting}
    $('.trigger').click(function() {
      $('.target').submit();
    });
    \end{lstlisting}
\end{itemize}
\end{frame}

\subsection{Eventos del Teclado} % (fold)

\begin{frame}[fragile]{Eventos del Teclado} % (fold)
\begin{itemize}
    \item .keypress(handler) \textit{ handler= Una función que se ejecutará cada vez que se dispare el evento.}
    \begin{lstlisting}
    $('.target').keypress(function() {
      $(this).log('Keypress event was triggered.');
    });
    \end{lstlisting}
    \item .keydown().
    \item .keyup().
\end{itemize}
\end{frame}

\subsection{Eventos del Navegador} % (fold)

\begin{frame}[fragile]{Eventos del Navegador} % (fold)
\begin{itemize}
    \item .resize(handler) \textit{ handler= Una función que se ejecutará cada vez que se dispare el evento.}
    \begin{lstlisting}
    $(window).resize(function() {
      $(this).log('Resize event was triggered.');
    });
    \end{lstlisting}
    \item .scroll(handler) \textit{ handler= Una función que se ejecutará cada vez que se dispare el evento.} 
    \begin{lstlisting}
    $('.trigger').click(function() {
      $('.target').scroll();
    });
    \end{lstlisting}
\end{itemize}
\end{frame}

\section{Métodos de Efecto} % (fold)

\subsection{Efectos Pre-empaquetados} % (fold)

\begin{frame}[fragile]{Efectos Pre-empaquetados} % (fold)
\begin{itemize}
    \item .show([speed][, callback]) \textit{ speed= una cadena o un número para determinar cuánto tiempo se ejecutara la animación.}

    \textit{callback= una función para saber cuando ha finalizado la animación.}
    \begin{lstlisting}
    <div class="content">
      <div class="trigger button">Trigger</div>
      <div class="target"><img src="hat.gif" width="80" height="54"></div>
      <div class="long"></div>
    </div>
    \end{lstlisting}
    \begin{lstlisting}
    $('.trigger').click(function() {
      $('.target').show('slow',  function() {
        $(this).log('Effect Complete.');
      });
    });
    \end{lstlisting}
    \item .hide().
    \item .toggle().
    \item .slideDown() .slideUp() .slideToggle()
    \item .fadeIn() .fadeOut() .fadeTo()
\end{itemize}
\end{frame}

\subsection{Efectos Personalizados} % (fold)

\begin{frame}[fragile]{Efectos Personalizados} % (fold)
\begin{itemize}
    \item .animate(properties [, speed][, easing][, callback])  \textit{
    properties= un mapa de propiedades CSS por los que se moverá la animación.}

    \textit{ speed= una cadena o un número para determinar el tiempo que se ejecutara la animación.}

    \textit{ easing= una cadena facilita el uso de la función en la transición.}

    \textit{ callback= una función para saber cuando ha finalizado la animación.}
    \begin{lstlisting}
    <div class="content">
      <div class="trigger button">Trigger</div>
      <div class="target"><img src="hat.gif" width="80" height="54"></div>
      <div class="long"></div>
    </div>
    \end{lstlisting}
    \begin{lstlisting}
    $('.trigger').click(function () {
      $('.target').animate({
      'width': 300, 
      'left': 100, 
      'opacity': 0.25
      }, 'slow', function() {
        $(this).log('Effect Complete');
      });  
    });
    \end{lstlisting}
\end{itemize}
\end{frame}

\section{Métodos AJAX} % (fold)

\subsection{Interfaz de Bajo Nivel} % (fold)

\begin{frame}[fragile]{Interfaz de Bajo Nivel} % (fold)
\begin{itemize}
    \item .ajax(settings) \textit{ settings= un mapa que contiene las siguientes opciones para loas solicitudes:}
    \begin{itemize}
      \item url.
      \item dataType.
      \item timeout.
      \item error.
      \item success.
    \end{itemize}
    \begin{lstlisting}
    $.ajax({
      url: 'ajax/test.html', 
      success: function(data) {
        $('.result').html(data);
        $().log('Load was performed');
      }, 
    });
    \end{lstlisting}
    \item .ajaxSetup(settings) \textit{ settings= un mapa de opciones para
    solicitudes a futuro.} 
    \begin{lstlisting}
    $.ajaxSetup({
      url: 'ping.php', 
    });
    \end{lstlisting}
\end{itemize}
\end{frame}

\subsection{Métodos de Taquigrafía} % (fold)

\begin{frame}[fragile]{Métodos de Taquigrafía} % (fold)
\begin{itemize}
    \item .get(url[, data][, success]) \textit{ url= una cadena que contiene la URL donde la solicitud debe ser enviada.}

    \textit{ data= un mapa de datos enviados con la solicitud.}

    \textit{ success= una función que se ejecuta cuando la solicitud a sido procesada.}
    \begin{lstlisting}
    $.get('ajax/test.html', function (data) {
      $('.result').html(data);
      $().log('Load was performed');
    })
    \end{lstlisting}
    \item .getModified().
    \item .loadModified().
    \item .post().
\end{itemize}
\end{frame}

\begin{frame}[fragile]{Métodos de Taquigrafía} % (fold)
\begin{itemize}
    \item .load(url[, data][, success]) \textit{ url= una cadena que contiene la URL donde la solicitud debe ser enviada.}

    \textit{ data= un mapa de datos enviados con la solicitud.}

    \textit{ success= una función que se ejecuta cuando la solicitud a sido procesada.}
    \begin{lstlisting}
    $('.result').load('ajax/test.html');
    \end{lstlisting}
    \item .getJSON(url[, data][, success]) \textit{ url= una cadena que contiene la URL donde la solicitud debe ser enviada.}

    \textit{ data= un mapa de datos enviados con la solicitud.}

    \textit{ success= una función que se ejecuta cuando la solicitud a sido procesada.}
    \begin{lstlisting}
    $.getJSON('ajax/test.json', function(data) {
      $('.result').html('<p> + data.foo + '</p><p>' + data.baz[1] + '</p>');
      $().log('Load was performed');
    });  
    \end{lstlisting}
    \item .getScript().
\end{itemize}
\end{frame}

\subsection{Controladores Globales de Eventos} % (fold)

\begin{frame}[fragile]{Controladores Globales de Eventos} % (fold)
\begin{itemize}
    \item .ajaxComplete(handler) \textit{ handler= la función que se invoca.}
    \begin{lstlisting}
    $('.log').ajaxComplete(function() {
      $(this).log('Triggered ajaxComplete handler');
    });
    \end{lstlisting}
    \item .ajaxError().
    \item .ajaxStart().
    \item .ajaxStop().
    \item .ajaxSuccess().
    \item .ajaxSend(handler) \textit{ handler= la función que se invoca.} 
    \begin{lstlisting}
    $('.log').ajaxSend(function() {
      $(this).log('Triggered ajaxSend handler');
    });
    \end{lstlisting}
\end{itemize}
\end{frame}

\subsection{Funciones de Ayuda} % (fold)

\begin{frame}[fragile]{Funciones de Ayuda} % (fold)
\begin{itemize}
    \item .serialize(param)
    \begin{lstlisting}
    <form>
      <div><input type="text" name="a" value="1" id="a" /></div>
      <div><input type="text" name="b" value="2" id="b" /></div>
      <div><input type="text" name="c" value="3" id="c" /></div>
      <div><textarea name="d" rows="8" cols="40">4</textarea></div>
      <div><select name="e">
        <option value="5" selected="selected">5</option>
        <option value="6">6</option><option value="7">7</option>
      </select></div>
      <div><input type="checkbox" name="f" value="8" id="f" /></div>
      <div><input type="submit" name="g" value="Submit" id="g" /></div>
    </form>
    \end{lstlisting}
    \begin{lstlisting}
    $('form').submit(function() {
      $(this).log($('input, textarea, select').serialize());
      return false;
    });
    \end{lstlisting}
\end{itemize}
\end{frame}

\section{Otros Métodos} % (fold)

\subsection{Otros Métodos} % (fold)

\begin{frame}[fragile]{Métodos de Codificación y decodificación base-64} % (fold)

\begin{itemize}
\item .atob() \textit{Método que decodifica una cadena codificada en base-64.}

\item .btoa() \textit{Método para codificar una cadena en base-64. Este método utiliza los caracteres A-Z,  a-z, 0-9, +, / y = para la codificación de una cadena.}
\end{itemize}

\textbf{Ejemplo}
\begin{lstlisting}
<html>
<head>
    <script type="text/javascript">
        function EncodeDecode () {
            input = document.getElementById ("myInput");

            encodedData = window.btoa (input.value);
            alert ("encoded data: " + encodedData);

            decodedData = window.atob (encodedData);
            alert ("decoded data: " + decodedData);
        }
    </script>
</head>
<body>
    <input type="text" id="myInput" value="The text to encode"/>
    <button onclick="EncodeDecode ();">Encode in base-64!</button>
</body>
</html>
\end{lstlisting}
\end{frame}

\begin{frame}[fragile]{JSON.parse} % (fold)
Método para convertir un array en Objeto.
\begin{itemize}
\item Usando JavaScript
\textbf{Ejemplo}
\begin{lstlisting}
var JsonString = '[{"name": "Jhon", "apellido":"Smith"}]';
var JsonObjects = JSON.parse(JsonString);
alert(JsonObjects);
\end{lstlisting}
\item Usando JQuery
\textbf{Ejemplo}
\begin{lstlisting}
var obj = jQuery.parseJSON('{"name": "Jhon", "apellido":"Smith"}');
alert(obj.name === "Jhon");
\end{lstlisting}
\end{itemize}
\end{frame}

\end{document}
