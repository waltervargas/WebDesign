% Copyleft 2010 by Walter Vargas <walter@covetel.com.ve>

\documentclass{beamer}
\usepackage{listings}
\lstset{
    language=HTML 
}
\lstset{
    basicstyle=\scriptsize, 
    stringstyle=\ttfamily,
    showstringspaces=false, 
    numbers=left,
    numberstyle=\scriptsize,
    tabsize=2,
}

% Configuración de la apariencia. 

\usetheme{Szeged}
\usecolortheme{beaver}
\usefonttheme[onlylarge]{structurebold}
\setbeamerfont*{frametitle}{size=\normalsize,series=\bfseries}

\usepackage{color}
\definecolor{lightgray}{rgb}{.9, .9, .9}
\definecolor{darkgray}{rgb}{.4, .4, .4}
\definecolor{purple}{rgb}{0.65,  0.12,  0.82}
\definecolor{azulito}{HTML}{0066CC}

\lstdefinelanguage{HTML}{
keywords={xml, DOCTYPE, head,  body, html}, 
keywordstyle=\color{green}\bfseries, 
ndkeywords={title, option, form, textarea, table, th, tr, td, p, br, input,
button, frame, iframe, div, optgroup, select, fieldset,  legend}, 
ndkeywordstyle=\color{blue}\bfseries, 
identifierstyle=\color{black}, 
sensitive=false, 
comment=[l]{//}, 
morecomment=[s]{/*}{*/}, 
stringstyle=\color{azulito}\ttfamily, 
morestring=[b]', 
morestring=[b]"
}

\lstset{
language=HTML, 
backgroundcolor=\color{lightgray}, 
extendedchars=true, 
basicstyle=\footnotesize\ttfamily, 
showstringspaces=false, 
showspaces=false, 
numbers=left, 
numberstyle=\footnotesize, 
numbersep=9pt, 
tabsize=2, 
breaklines=true, 
showtabs=false, 
captionpos=b 
}


% Paquetes
\usepackage[utf8]{inputenc}
\usepackage[spanish]{babel}

% Titulo
\title[WebDesign] {Diseño Web | Tecnologías Involucradas}
\author[Walter Vargas]{ info@covetel.com.ve \inst{1}}
\subtitle{Formularios}
\institute[covetel.com.ve]{ \inst{1} Cooperativa Venezolana de Tecnologías Libres R.S. }
\date

\begin{document}


\begin{frame}
  \titlepage
\end{frame}

\begin{frame}{Tabla de Contenidos} % (fold)
\tableofcontents
\end{frame}

\section{Formularios}

\subsection{Elementos de un formulario} % (fold)

\begin{frame}{Elementos de un Formulario} % (fold)
    \begin{center}
        \begin{itemize}
            \item \texttt{form} Establece el formulario
            \item \texttt{input} Crea distintos controles 
            \item \texttt{button} Botón genérico de entrada. 
            \item \texttt{textarea} Control de entrada de texto multilinea. 
            \item \texttt{select} Menu de elección múltiple o lista deslizante.
            \item \texttt{option} Una opción dentro de un control
            \texttt{select}
            \item \texttt{optgroup} Define un grupo de opciones
            \item \texttt{label} Adjunta información de los controles
            \item \texttt{fieldset} Agrupa controles y etiquetas relacionados.
            \item \texttt{legend} Asigna una leyenda a un conjunto de campos.
        \end{itemize}
    \end{center}
\end{frame}

\begin{frame}{Consideraciones de un formulario} % (fold)
    \begin{center}
        El formulario puede tener cualquier contenido Web (imágenes, texto,
        tablas) pero su función es la de contenedor de distintos controles
        utilizados para introduccir información. \\[0.4cm]
        \pause

        Puede tener varios formularios en un solo documento pero no pueden
        anidarse, y debe evitar que se superpongan. \\[0.4cm]
        \pause

        Cuando un formulario tiene más de 4 grupos de campos, debe considerar
        utilizar el patrón de diseño \textit{Process Funnel}\\[0.4cm]
        \pause

        Cuando un usuario rellena un formulario y pulsa el botón de envío el
        navegador toma la información,  la dispone en parejas nombre/valor, la
        codifica para la transferencia y la envía al servidor. 
    \end{center}
\end{frame}

\subsubsection{form} 

\begin{frame}{El elemento básico de un formulario \texttt{<form>} } % (fold)
    \textbf{Atributos}
    \begin{itemize}
        \item \textit{Núcleo} (\texttt{id, class, style, title}),
        \textit{Internacionalización}, \textit{Eventos}, \texttt{onsubmit,
        onreset, onblur}
        \item \texttt{accept="lista-tipo-contenido"}
        \item \texttt{acept-charset="lista-conjunto-caracteres"}
        \item \texttt{action="url"}
        \item \texttt{enctype="tipo contenido"}
        \item \texttt{method="get|post"}
        \item \texttt{name="texto"} (\textit{relegado} en XHTML en favor del
        atributo \texttt{id})
        \item \texttt{target="nombre"} 
    \end{itemize}
\end{frame}

\subsubsection{action} 

\begin{frame}[fragile]{El atributo \texttt{action}} % (fold)
\begin{center}
    
    El atributo \texttt{action} del elemento \texttt{form} proporciona la URL
    del programa del lado del servidor que se utilizará para procesar el
    formulario.
    
    \begin{lstlisting}
    <form action="/personas/guardar">

    </form>
    \end{lstlisting}

\end{center}
\end{frame}

\subsubsection{fragile} 

\begin{frame}[fragile]{El atributo \texttt{method}} % (fold)
\begin{center}
    
    El atributo \texttt{method} especifica uno de los dos métodos \texttt{get}
    o \texttt{post} para enviar la información del formulario al servidor.
    \\[0.4cm]
    \pause
    La información suele transferirse en una serie de variables separadas por
    el carácter \textit{ampersand} \texttt{\&} como se puede ver en el siguiente
    ejemplo: 

    \begin{verbatim}
    nombre=Walter&apellido=Vargas&ciudad=Caracas
    \end{verbatim}
    \pause
    El atributo \texttt{name} del los elementos de control del formulario
    proporciona el nombre de la variable. El contenido que el usuario
    introduce, constituye el contenido de la variable.

\end{center}
\end{frame}

\subsection{Métodos GET y POST} % (fold)

\begin{frame}[fragile]{Envío via GET \texttt{method="get"}} % (fold)
    \begin{center}
        Con el método \texttt{get}, el navegador transfiere los datos del
        formulario como parte de la propia URL (añadidos al final y separados
        por un signo de interrogación) en una única transmisión. 

        \begin{verbatim}
REQUEST_URI: /cgi-bin/headers.pl?nombre=walter&apellido=vargas&ciudad=caracas
        \end{verbatim}
    \end{center}
\end{frame}

\subsubsection{Cabeceras get} 

\begin{frame}[fragile]{Cabeceras \texttt{get}} % (fold)
    \begin{center}
    \footnotesize{ 
        \begin{verbatim}
GET /cgi-bin/headers.pl?nombre=Walter&apellido=Vargas&ciudad=caracas HTTP/1.1
Host: localhost
User-Agent: Mozilla/5.0 (Macintosh; Intel Mac OS X 10.6; rv:6.0.2) Gecko/20100101 Firefox/6.0.2
Accept: text/html,application/xhtml+xml,application/xml;q=0.9,*/*;q=0.8
Accept-Language: es-ve,es-es;q=0.8,en;q=0.5,en-us;q=0.3
Accept-Encoding: gzip, deflate
Accept-Charset: ISO-8859-1,utf-8;q=0.7,*;q=0.7
Connection: keep-alive
Referer: http://localhost/form.html
        \end{verbatim}
        }
    \end{center}

\end{frame}

\begin{frame}{Envío via \texttt{method="post"}} % (fold)
    \begin{center}
        El método \texttt{post} transmite la información de entrada del
        formulario separada de la URL, en un mensaje dividido en dos partes. La
        primera parte del mensaje es solo la cabecera especial enviada por el
        navegador con cada solicitud. \\[0.4cm]
        \pause
        Cuando el servidor ve la cadena \textit{post} al comienzo del mensaje,
        queda en escucha de los datos. 
    \end{center}
\end{frame}

\subsubsection{Cabeceras post} 

\begin{frame}[fragile]{Cabeceras \texttt{post}} % (fold)

\footnotesize{
\begin{verbatim}
POST /cgi-bin/headers.pl HTTP/1.1
Host: localhost
User-Agent: Mozilla/5.0 (Macintosh; Intel Mac OS X 10.6; rv:6.0.2) Gecko/20100101 Firefox/6.0.2
Accept: text/html,application/xhtml+xml,application/xml;q=0.9,*/*;q=0.8
Accept-Language: es-ve,es-es;q=0.8,en;q=0.5,en-us;q=0.3
Accept-Encoding: gzip, deflate
Accept-Charset: ISO-8859-1,utf-8;q=0.7,*;q=0.7
Connection: keep-alive
Referer: http://localhost/form.html
Content-Type: application/x-www-form-urlencoded
Content-Length: 44    
\end{verbatim}

\begin{verbatim}
Line-based text data: application/x-www-form-urlencoded
    nombre=Walter&apellido=Vargas&ciudad=caracas    
\end{verbatim}
}
\end{frame}


\subsection{Codificación} % (fold)

\begin{frame}{Codificación \texttt{<form enctype="valor">}} % (fold)
    \begin{center}
        Otro paso de la transacción es la codificación de los datos.\\
        Es un método para traducir espacios y otros caracteres no permitidos en
        las URL en sus equivalentes hexadecimales. Por ejemplo el espacio se
        traduce como \texttt{\%20} y la barra inclinada como \%2F.\\[0.4cm] 
        \pause

        El formato de codificación por defecto es
        \texttt{application/x-www-form-urlencoded}.\\[0.4cm] 
        \pause
        Si su formulario incluye adjuntos (\texttt{file}) para cargar archivos
        al servidor, debe utilizar \texttt{multipart/form-data}
    \end{center}
\end{frame}

% subsection Codificación tt (end)

\subsection{Controles de Formulario} % (fold)

\begin{frame}{Controles de Entrada} % (fold)
    El elemento \texttt{input} se utiliza para crear distintos controles de
    entrada de formulario,  entre ellos: 

    \begin{itemize}
        \item Campos de entrada de texto de una sola linea
        \item Campos de entrada de contraseña
        \item Controles ocultos 
        \item Casillas de verificación 
        \item Botones de opción
        \item Botones de envío y de reinicio
        \item Mecanismos de subida de archivos
        \item Botones de imagen y personalización
    \end{itemize}

    El atributo \texttt{type} del elemento \texttt{input} especifica el tipo de
    control. 
\end{frame}

\subsubsection{Entrada de Texto}

\begin{frame}{\texttt{Entrada de Texto <input type='text'>}} % (fold)
    \textbf{Atributos}
    \begin{itemize}
        \item Núcleo (\texttt{id, class, style, title}) Internacionalización,
        Eventos, Foco (\texttt{accesskey,tabindex, onfocus, onblur})
        \item  \texttt{disabled='disabled'}
        \item \texttt{maxlenght='número'}
        \item \texttt{name='text'} (\textit{requerido})
        \item \texttt{readonly='readonly'}
        \item \texttt{size='número'}
        \item \texttt{value='valor'}
    \end{itemize}
\end{frame}

\subsubsection{Entrada de Contraseña}

\begin{frame}{Entrada de contraseña \texttt{<input type='password'>}} % (fold)
    \textbf{Atributos}
    \begin{itemize}
        \item Núcleo (\texttt{id, class, style, title}) Internacionalización,
        Eventos, Foco (\texttt{accesskey,tabindex, onfocus, onblur})
        \item  \texttt{disabled='disabled'}
        \item \texttt{maxlenght='número'}
        \item \texttt{name='text'} (\textit{requerido})
        \item \texttt{readonly='readonly'}
        \item \texttt{size='número'}
        \item \texttt{value='valor'} (\textit{requerido})
    \end{itemize}

\end{frame}

\subsubsection{Entrada oculta}

\begin{frame}{Entrada oculta \texttt{<input type='hidden'>}} % (fold)
    \textbf{Atributos}
    \begin{itemize}
        \item  \texttt{accesskey='caracter'}
        \item \texttt{tabindex='número'}
        \item \texttt{name='texto'} (\textit{requerido})
        \item \texttt{value='valor'} (\textit{requerido})
    \end{itemize}
\end{frame}


\subsubsection{Casilla de Verificación} 

\begin{frame}[fragile]{Casilla de Verificación \texttt{<input type='checkbox'>}} % (fold)
    \textbf{Atributos}
    \begin{itemize}
        \item Núcleo (\texttt{id, class, style, title}) Internacionalización,
        Eventos, Foco (\texttt{accesskey,tabindex, onfocus, onblur})
        \item \texttt{align='left|right|top|texttop|middle|baseline|bottom'}
        \item \texttt{disabled='disabled'}
        \item \texttt{maxlenght='número'}
        \item \texttt{name='text'} (\textit{requerido})
        \item \texttt{readonly='readonly'}
        \item \texttt{size='número'}
        \item \texttt{value='valor'}
    \end{itemize}
\end{frame}


\begin{frame}[fragile]{Casilla de Verificación \texttt{<input type='checkbox'>}} % (fold)
\footnotesize{
    \begin{lstlisting}
<form>
<input type="checkbox" name="os" value="debian" /> I like Debian <br />
<input type="checkbox" name="os" value="arch" /> I like Arch <br /> 
<input type="checkbox" name="os" value="ubuntu" /> I like Ubuntu<br /> 
</form>
    \end{lstlisting}
}
\end{frame}

\subsubsection{Botón de opción} 
\begin{frame}{Botón de opción} % (fold)
    \textbf{Atributos}
    \begin{itemize}
        \item Núcleo (\texttt{id, class, style, title}) Internacionalización,
        Eventos, Foco (\texttt{accesskey,tabindex, onfocus, onblur})
        \item \texttt{chequed='chequed'}
        \item \texttt{disabled='disabled'}
        \item \texttt{name='text'} (\textit{requerido})
        \item \texttt{readonly='readonly'}
        \item \texttt{value='valor'}
    \end{itemize}
    
\end{frame}

\begin{frame}[fragile]{Botón de opción} % (fold)
\footnotesize{
\begin{lstlisting}
<form>
<input type="radio" name="sex" value="femenino" /> Femenino<br />
<input type="radio" name="sex" value="masculino" /> Masculino <br />
<input type="radio" name="sex" value="otro" /> Otro <br />
</form>    
\end{lstlisting}
}
\end{frame}

\subsubsection{Botones de envío y reinicio} 

\begin{frame}{Botón de envío \texttt{type='submit'}} % (fold)
    \textbf{Atributos}
    \begin{itemize}
        \item Núcleo (\texttt{id, class, style, title}) Internacionalización,
        Eventos, Foco (\texttt{accesskey,tabindex, onfocus, onblur})
        \item \texttt{disabled='disabled'}
        \item \texttt{name='text'} (\textit{requerido})
        \item \texttt{value='valor'}
    \end{itemize}
\end{frame}

\begin{frame}[fragile]{Botón de envío} % (fold)
    \begin{lstlisting}
<form name="input" action="/login" method="get">
    Username: <input type="text" name="user" />

    <input type="submit" value="Submit" />
</form>        
    \end{lstlisting}
\end{frame}

\subsubsection{Botón de reinicio}

\begin{frame}[fragile]{Botón de reinicio \texttt{type='reset'}} % (fold)
    \textbf{Atributos}
    \begin{itemize}
        \item Núcleo (\texttt{id, class, style, title}) Internacionalización,
        Eventos, Foco (\texttt{accesskey,tabindex, onfocus, onblur})
        \item \texttt{disabled='disabled'}
        \item \texttt{value='valor'}
    \end{itemize}
    
\end{frame}

\subsubsection{Botón personalizado }

\begin{frame}{Botón personalizado \texttt{type='button'}} % (fold)
    Puede crear un botón personalizado que sea utilizado con controles de
    \textit{scripting} (JavaScript) del lado del cliente ajustando el tipo
    \texttt{input} a \texttt{button}\\
    \footnotesize{
    \textbf{Atributos}
    \begin{itemize}
        \item Núcleo (\texttt{id, class, style, title}) Internacionalización,
        Eventos, Foco (\texttt{accesskey,tabindex, onfocus, onblur})
        \item \texttt{disabled='disabled'}
        \item \texttt{name='texto'}
        \item \texttt{value='texto'}
        \item \texttt{align='left|right|top|texttop|middle|baseline|bottom'}
    \end{itemize}
    }
\end{frame}

\subsubsection{Botón de imagen}

\begin{frame}[fragile]{Botón de imagen \texttt{type='image'}} % (fold)
    Si quiere utilizar su propia imagen para un boton de envío,  utilice el
    tipo \texttt{image}\\

    \footnotesize{
    \textbf{Atributos}
    \begin{itemize}
        \item Núcleo (\texttt{id, class, style, title}) Internacionalización,
        Eventos, Foco (\texttt{accesskey,tabindex, onfocus, onblur})
        \item \texttt{disabled='disabled'}
        \item \texttt{name='texto'} (\textit{texto requerido})
        \item \texttt{src='url'}
        \item \texttt{alt='texto'}
        \item \texttt{align='top|middle|bottom'}
    \end{itemize}
    }
\begin{lstlisting}
<input type="image" src="submit.gif" alt="Submit" />
\end{lstlisting}

\end{frame}

\subsubsection{Selector de archivos}
\begin{frame}{Selector de archivos \texttt{type='file'}} % (fold)
    El tipo de entrada \texttt{file} permite enviar archivos externos con el
    formulario. El control del formulario incluye un campo de texto y un botón
    de navegación que accede a los contenidos de la computadora local.\\ 
    
    \footnotesize{
    \textbf{Atributos}
    \begin{itemize}
        \item Núcleo (\texttt{id, class, style, title}) Internacionalización,
        Eventos, Foco (\texttt{accesskey,tabindex, onfocus, onblur})
        \item \texttt{disabled='disabled'}
        \item \texttt{name='texto'} (\textit{texto requerido})
        \item \texttt{accept='Tipo de MIME'}
        \item \texttt{maxlength='número'}
        \item \texttt{size='número'}
        \item \texttt{readonly='readonly'}
        \item \texttt{value='texto'}
    \end{itemize}
    }
\end{frame}

\subsubsection{Area de texto multilinea} 

\begin{frame}[fragile]{Área de texto multilinea \texttt{<textarea> </textarea>}} % (fold)
El elemento \texttt{textarea} crea una casilla de entrada de texto deslizable y
multilinea que permite introducir entradas de texto amplias. \\

\footnotesize{
    \begin{itemize}
        \item Núcleo (\texttt{id, class, style, title}) Internacionalización,
        Eventos, \texttt{onselect}, \texttt{onchange},  Foco (\texttt{accesskey,tabindex, onfocus, onblur})
        \item \texttt{cols='número'} (\textit{requerido})
        \item \texttt{disabled='disabled'}
        \item \texttt{name='text'} (\textit{requerido})
        \item \texttt{readonly='readonly'}
        \item \texttt{rows='número'} (\textit{requerido})
    \end{itemize}
    \begin{lstlisting}
<textarea rows="2" cols="20">
Neque porro quisquam est qui dolorem ipsum quia dolor sit amet, consectetur, adipisci velit...
</textarea>

    \end{lstlisting}
}
\end{frame}

\subsubsection{Listas de selección} 

\begin{frame}[fragile]{Listas de selección} % (fold)
    \footnotesize{
    El elemento \texttt{select} permite generar un menu de opciones más
    compacto que el que proporcionan las agrupaciones de casillas de
    verificación o de botones de opción. Un menu se muestra como un menu
    desplegable o como una lista deslizante de opciones, dependiendo del tamaño
    especificado. El elemento \texttt{select} funciona a modo de contenedo para
    varios elementos \texttt{opcion}. Tambien puede contener uno o más
    \texttt{optgroups}, utilizado para definir un grupo lógico de elementos
    \texttt{opcion}. 

\begin{lstlisting}
<select>
  <option value="volvo">Volvo</option>
  <option value="saab">Saab</option>
  <option value="mercedes">Mercedes</option>
  <option value="audi">Audi</option>
</select>    
\end{lstlisting}

    }
\end{frame}

\begin{frame}{\texttt{<select> </select>}} % (fold)
    \textbf{Atributos}
    \begin{itemize}
        \item Núcleo (\texttt{id, class, style, title}) Internacionalización,
        Eventos, \texttt{onfocus}, \texttt{onblur}, \texttt{onchange}
        \item \texttt{disabled='disabled'}
        \item \texttt{name='text'} (\textit{requerido})
        \item \texttt{multiple='multiple'} 
        \item \texttt{size='número'}
        \item \texttt{tabindex='número'}
    \end{itemize}
\end{frame}


\begin{frame}{\texttt{<option> </option>}} % (fold)
    \textbf{Atributos}
    \begin{itemize}
        \item Núcleo (\texttt{id, class, style, title}) Internacionalización,
        Eventos
        \item \texttt{disabled='disabled'}
        \item \texttt{label='texto'}
        \item \texttt{selected='selected'}
        \item \texttt{value='texto'}
    \end{itemize}
\end{frame}

\begin{frame}[fragile]{\texttt{<optgroup> </optgroup>}} % (fold)
    \textbf{Atributos}
    \begin{itemize}
        \item Núcleo (\texttt{id, class, style, title}) Internacionalización,
        Eventos
        \item \texttt{disabled='disabled'}
        \item \texttt{label='texto'}
    \end{itemize}

    \begin{lstlisting}
<select>
  <optgroup label="Swedish Cars">
    <option value="volvo">Volvo</option>
    <option value="saab">Saab</option>
  </optgroup>
  <optgroup label="German Cars">
    <option value="mercedes">Mercedes</option>
    <option value="audi">Audi</option>
  </optgroup>
</select>        
    \end{lstlisting}
    
\end{frame}

\begin{frame}[fragile]{Listas de selección multiples} % (fold)
    Para que la lista sea de selección multiple, especifique el número de
    lineas que le gustaría que fuerán visibles en la lista utilizando el
    atributo \texttt{size} o añada el atributo \texttt{multiple} al elemento
    \texttt{select}. 

    \begin{lstlisting}
<select multiple="multiple" size="2">
  <option value="volvo">Volvo</option>
  <option value="saab">Saab</option>
  <option value="mercedes">Mercedes</option>
  <option value="audi">Audi</option>
</select>
    \end{lstlisting}

\end{frame}

\subsubsection{Botones}

\begin{frame}[fragile]{Botones \texttt{<button type='button'> </button>}} % (fold)
\footnotesize{
    El elemento \texttt{button} define un 'botón' personalizado que funciona
    como los botones creados con la etiqueta \texttt{input}. El elemento
    \texttt{button} puede tener imagenes (pero no mapas de imágen) y cualquier
    otro contenido excepto elementos \texttt{a}, \texttt{form} y elementos de
    control de formulario. Los botones pueden renderizarse como botones 3D
    sombreados, con movimiento arriba y abajo cuando se pulsan (como los
    botones de envío y de reinicio), a diferencia del tipo de entrada
    \texttt{image} que es solo una imágen plana. 

    \begin{lstlisting}
<button type="button">Click Me!</button>
    \end{lstlisting}

}
\end{frame}

\begin{frame}{Botones lista de atributos} % (fold)
    \begin{itemize}
        \item Núcleo (\texttt{id, class, style, title}) Internacionalización,
        Eventos, Foco (\texttt{acceskey} \texttt{tabindex} \texttt{onfocus}
        \texttt{onblur})
        \item \texttt{disabled='disabled'}
        \item \texttt{name='texto'}
        \item \texttt{type='submit|reset|button'}

    \end{itemize}
\end{frame}

\subsubsection{Fieldset y Legend} 
\begin{frame}{Fieldset y Legend} % (fold)

\footnotesize{
El elemento \texttt{fieldset} se utiliza para crear un grupo lógico de
controles de formulario. El \texttt{fieldset} puede contener un elemento
\texttt{legend}, una descripción de los campos incluidos que puede resultar muy
útil a los navegadores no visuales y lectores de pantalla.\\[0.5cm] 

    \textbf{Atributos de \texttt{fieldset}}
    \begin{itemize}
        \item Núcleo (\texttt{id, class, style, title}) Internacionalización,
        Eventos \\[0.5cm]
    \end{itemize}
    \textbf{Atributos de \texttt{legend}}
    \begin{itemize}
        \item Núcleo (\texttt{id, class, style, title}) Internacionalización,
        Eventos
        \item \texttt{acceskey='caracter'}
        \item \texttt{align='top|bottom|left|right'} (\textit{relegado})
    \end{itemize}

} 

\end{frame}

\begin{frame}[fragile]{Fieldset y Legend} % (fold)
\begin{lstlisting}
<form>
  <fieldset>
    <legend>Personalia:</legend>
    Name: <input type="text" size="30" /><br />
    Email: <input type="text" size="30" /><br />
    Date of birth: <input type="text" size="10" />
  </fieldset>
</form>
\end{lstlisting}
\end{frame}


\begin{frame}{\texttt{acceskey}} % (fold)

\begin{center}

    El atributo \texttt{accesskey} especifica un carácter que será usado como
    atajo de teclado para un elemento. La funcionalidad real de la tecla de
    acceso puede necesitar de una combinación de tecla como \texttt{Alt-tecla}
    (Windows) o \texttt{Comando-Tecla} para Mac. \\[0.5cm] 
    \pause
        El atributo \texttt{accesskey} puede utilizarse con los elementos de
    control de formulario \texttt{button},  \texttt{input},  \texttt{label},
    \texttt{legend} y \texttt{textarea}. 

\end{center}

\end{frame}


\begin{frame}{\texttt{tabindex}} % (fold)
    \begin{center}
        Utilice \texttt{tabindex} si quiere reorganizar el orden del foco sin
        cambiar el orden de los elementos.
    \end{center}
\end{frame}

% subsection Controles de Formulario (end)



\end{document}
